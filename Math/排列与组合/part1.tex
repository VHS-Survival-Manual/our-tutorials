\section{\textbf{计数原理}}

{\large 分类计数原理}~~完成一件事,有$n$类办法,在第1类办法中有$m_1$种不同的办法,在第$n$类办法中有$m_n$种不同的办法,那么完成这件事共有$\mathbf{N}m_{1}+m_{2}+\cdots{}+m_{n}$种不同的办法。

{\large 分步计数原理}~~完成一件事,有$n$类办法,在第1类办法中有$m_1$种不同的办法,在第$n$类办法中有$m_n$种不同的办法,那么完成这件事共有$\mathbf{N}m_{1}\cdot{}m_{2}\cdot{}\cdots{}\cdot{}m_{n}$种不同的办法。

\subsection{例题}

{\color{blue} 例一:}乔布斯有4个苹果和3个菠萝,库克要选择一种水果,问有多少种选择

{\color{blue} 解答:}$4+3=7$

{\color{blue} 例二:}乔布斯有4个苹果和3个菠萝以及5个柚子,库克要选择两种水果,问有多少种选择

{\color{blue} 解答:}$4\times{}3\times{}5=60$

\section{\textbf{排列}}

\subsection{排列问题}

我们先从\textbf{排列问题}开始学习。

假设水果篮中有苹果、香蕉、橙子、葡萄和芒果五种不同的水果,若从中选取两种水果按特定顺序排列(例如「苹果→香蕉」和「香蕉→苹果」视为两种不同的排列),则不同的顺序会直接影响排列结果。例如,从5种水果中选2种排列时,排列数为$\textbf{P}(5,~2) = 5 \times 4 = 20$种

\textbf{——}这类似于将水果按不同顺序摆成果盘(如香蕉在前或苹果在前),每种顺序代表独特的外观。

\begin{enumerate}
    \item \textbf{有序性:}如同水果在果盘中的位置(首位放苹果或首位放葡萄)直接影响最终形态;
    \item \textbf{无序性:}一种水果不能同时占据两个位置(例如苹果不能既在首位又在次位)。
\end{enumerate}

而组合问题则类似「水果盲盒」:只需关注选中的水果种类(如苹果和葡萄),不关心被选中的顺序(苹果先放或葡萄先放均视为同一种组合)。

\subsubsection{排列问题概括}

我们把被选取的对象(上面的水果中的任意一个)叫做元素。那么上面的问题就是从五个不同的元素中任选两个元素,然后按照一定的顺序排成一列,从$\mathit{n}$个不同元素中任取$m(m \le n)$个元素,按照一定的顺序排成一列,叫作从$\mathit{n}$个不同元素中任取$\mathit{m}$个元素的一个\textbf{排列}。

\subsection{排列数公式}
这里的概念来自\textbf{教科书\footnote{即北师大出版的中职学校公共基础课程教材}}上,我们直接在下面引用。

一般地,从$\mathit{n}$个不同元素中任取$m(m \le n)$个元素的所有排列的个数,叫做从$\mathit{n}$个不同元素中任取$\mathit{m}$个元素的\textbf{排列数},用符号$\textbf{A}_{n}^{m}$表示。

很简单,就是有5个水果然后要任选2种水果,并求出共可以组成多少个没有重复组合的两种水果。简而言之就是求在$\textbf{m}$中任取$\textbf{n}$个水果共可以有多少种无重复水果组合。

但在这之前我们来了解下什么是\textbf{阶乘}

\subsubsection{阶乘}\label{sec:factorial}
阶乘,是数学中的一个基本概念,表示为$n!$,它定义为所有小于及等于$n$的正整数的乘积。具体来说,对于一个非负整数$n$,其阶乘$n!$计算方式是从1开始,一直乘到$n$,即:
\begin{equation*}
    n! = 1 \times 2 \times 3 \times \cdots \times n
\end{equation*}

没错,又是~AI~生成。但这个真的很简单,相信聪明的你很快就理解了。在大多数情况下十位数的阶乘往往需要使用计算器来辅助计算,如:$12!$,你能快速地算出结果吗?我觉得你不是电脑,答案是\textbf{479001600},我相信没有人能算的这么快,一般建议大家熟记前十位阶乘的结果,这就够了。

\begin{flushleft}
	{\color{red} 注意:}我们定义\textbf{$0! = 1$(零的阶乘等于一)}
\end{flushleft}

\noindent\hfil$*$\hfil$*$\hfil$*$\hfil \\

\textbf{那怎么求 $\textbf{A}_{n}^{m}$ 呢?} 

我们知道每一个排列都是从$n$个不同元素中任取$m(m \le n)$个元素,按照顺序排成一列的,所以我们可以把每个排列看成从$n$个不同元素中取$m$次,当然,拿都拿了就不会放回去了。就这样,计算它的排列数可以分$m$步完成。

根据分布计数原理,从$n$个不同元素中取出$m$个元素的排列数即为
\begin{equation*}
    \textbf{A}_{n}^{m} = n(n-1)(n-2)*\cdots{}*(n-m+1)
\end{equation*}
在这个公式中,右边是从正整数$n$开始的$n$个连续的正整数相乘,即从正整数$1$到$n$的连续积,这个连续积就是上面的\textbf{阶乘(详情:\ref{sec:factorial})}。

所以,从$n$个不同元素的全排列数公式为
\begin{equation*}
    \textbf{A}_{n}^{n}=n!~=n(n-1)(n-2)*\cdots{}*3*2*1
\end{equation*}
因为$(n-m)!~=(n-m)(n-m-1)*(n-m-2)*\cdots{}*3*2*1$,所以排列数公式还可以写成
\begin{equation}
    \textbf{A}_{n}^{m}=\frac{n!}{(n-m)!}
\end{equation}

还记得上面我们关于\textbf{阶乘}的定义吗?$0!~=~1$,所以,当$n=m$时,$(n-m)!~=0!~=1$,上述公式依然成立

\subsubsection{例题}
{\color{blue} 例一:}{请计算$\textbf{A}_{5}^{2}$}~与~$\textbf{A}_{11}^{3}$ 的值。

{\color{blue} 解答:}$\textbf{A}_{5}^{2} = 5 \times 4 = 20$,$\textbf{A}_{11}^{3}=11 \times 11 \times 10 \times 9 = 990$

在这里就不给太多例题了,这一章的重点还在后面。下一节就是\textbf{组合},如果上面的东西你学会了那么下一节就不会有任何问题。
