\part{\textbf{拓展部分}}

\section{\textbf{二项式定理}}\label{sec:binomial}

这里就直接放公式了,关于二项式定理的历史大家可以自行搜索。
\begin{equation}
    (a+b)^n = {\textstyle \sum_{k=0}^{n}} \binom{n}{k} a^{n-k}b^k
\end{equation}

当$n$为整数、其主要代表系数时用$\textbf{C}$形式,当$n$拓展为实数$\alpha$、$\alpha$和$k$是运算等相对复杂时只能括号。请大家在后面的学习中一定要注意它们之间的差异。

\section{求和符号}

\section{求积符号}

我们已经讲过了阶乘,然后我们用一个全新的符号来表示阶乘:
\begin{equation*}
    n! = \prod_{i=1}^{n}i
\end{equation*}

\section{双阶乘}

两个\emph{感叹号}就是双阶乘,它是这样表示的:$n!!$。我们已经知道了单个阶乘符号的运算方法,那么两个该如何计算?一般地, 对自然数 $n$,从 $n$ 开始逐次减 $2$ 的所有正整数的乘积称为 $n$ 的双阶乘(double factorial),记作:
$n!! = $ \begin{cases}
  n \times (n-2) \times (n-4) \times \cdots \times 6 \times 4 \times  2 \\
  n \times (n-2) \times (n-4) \times \cdots \times 5 \times 3 \times  1
\end{cases}

同时我们也约定 $0!! = 1,~(-1)! = 1$
% https://wuli.wiki/online/factor.html
% https://wuli.wiki/online/ProdSy.html

\section{Gamma 函数}