\section{\textbf{排列与组合的应用}}

排列与组合的应用广泛且深入多个领域,其核心在于处理元素的选择与顺序问题。

排列组合问题与组合问题的区别;
\begin{center}
    排列问题 \~ 与顺序有关; \\
    组合问题 \~ 与顺序无关
\end{center}

排列问题与组合问题的联系:\\
排列问题实际上是先从$n$个不同元素中取出$m$个元素(组合问题),再把这$m$个元素全排列。\\
排列数与组合数的关系:\\
\begin{equation}
    \textbf{A}_{n}^m = \textbf{C}_{m}^n \cdot \textbf{A}_{m}^{m}
\end{equation}

\subsection{例题(待审阅)}
% 强烈的怨念
在“网络建设与运维”技能大赛中,共有20支队伍,可是比赛场地只有四间教室,在抽取比赛顺序时将20支队伍分成四个小组,小组内采用单循环赛制(每两队比赛一场)决出前两名参加国赛。请问共需比赛多少场?

{\color{blue} 解答} $2\times{}\textbf{C}_{8}^{2}+\textbf{A}_{4}^{2}$

\subsection{举例}

把甲、乙、丙3名学生中分3次选取,每次选举1名学生的结果如表所示。

\begin{center}
\begin{tabular}{|c|c|c|}
	\hline
	第一次选举 & 第二次选举 & 第三次选举 \\
	\hline
	甲 & 乙 & 丙 \\
	甲 & 丙 & 乙 \\
	乙 & 甲 & 丙 \\
	乙 & 丙 & 甲 \\
	丙 & 甲 & 乙 \\
	丙 & 乙 & 甲 \\
	\hline
\end{tabular}
\end{center}

也就是说,从3个不同的元素中分3次选取,每次选取1个个元素的组合数与顺序有关的,选取的顺序影响了结果,使分组结果出现了重复,重复数是$\textbf{A}_3^3$。

故把甲、乙、丙3名学生分成3组,每组1个人,不同的分组方法共有$\frac{\textbf{C}_3^{1} + \textbf{C}_2^{1} + \textbf{C}_1^{1}}{\textbf{A}_3^3}=1$种

将$n$个不同的元素平均分成$m$堆(每堆$m$个)。若求有多少种方法可以先分$m$次计算组合数,再除以$m$个元素的全排列,其计算公式为
\begin{equation}
	\frac{\textbf{C}_{n}^{k} \cdot{} \textbf{C}_{n-k}^{k} \cdot ~\cdots ~\cdot \textbf{C}_{k}^{k}}{\textbf{A}_{m}^m}~(n=mk,~m,~n,k\in{}\textbf{N}_+)
\end{equation}
