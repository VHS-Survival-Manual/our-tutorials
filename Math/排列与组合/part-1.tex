\part{\textbf{排列}}

\section{排列问题}

什么是排列问题?我们先列举一个例子来给大家说明一下。假设水果篮中有苹果、香蕉、橙子、葡萄和芒果
五种不同的水果,若从中选取两种水果按特定顺序排列(例如「苹果→香蕉」和「香蕉→苹果」视为两种不同的
排列),则不同的顺序会直接影响排列结果。例如,从5种水果中选2种排列时,
排列数为$\textbf{A}_{5}^2 = 5 \times 4 = 20$种,,下面是它们的两个性质:

\begin{enumerate}
    \item \textbf{有序性:}如同水果在果盘中的位置(首位放苹果或首位放葡萄)直接影响最终形态;
    \item \textbf{无序性:}一种水果不能同时占据两个位置(例如苹果不能既在首位又在次位)。
\end{enumerate}

而组合问题则类似「水果盲盒」:只需关注选中的水果种类(如苹果和葡萄),不关心被选中的顺序(苹果先放或葡萄先放均视为同一种组合)。大家可以在本子上画几个点

\subsection{排列问题概括}

我们把被选取的对象(上面水果中的任意一个)叫做元素。那么上面的问题就是从五个不同的元素中任选两个元素,然后按照一定的顺序排成一列,从$\mathit{n}$个不同元素中任取$m(m \le n)$个元素,按照一定的顺序排成一列,叫作从$\mathit{n}$个不同元素中任取$\mathit{m}$个元素的一个\textbf{排列}。

\subsection{阶乘}\label{sec:factorial}
阶乘,是数学中的一个基本概念,表示为$n!$,它定义为所有小于及等于$n$的正整数的乘积。具体来说,对于一个非负整数$n$,其阶乘$n!$计算方式是从1开始,一直乘到$n$,即:
\begin{equation*}
    n! = 1 \times 2 \times 3 \times \cdots \times n
\end{equation*}
或者
\begin{equation*}
    n! = n \times (n-1) \times (n-2) \times \cdots \times 2 \times 1
\end{equation*}

在大多数情况下十位数以上的的阶乘往往需要使用计算器来辅助计算,如:$12!$,你能快速地算出结果吗?,答案是\textbf{479001600},我相信没有人能算的这么快,一般建议大家熟记前十位阶乘的结果,这就够了。

{\color{red} 注意:}我们定义\textbf{$0! = 1$(零的阶乘等于一)},当然这只是关于阶乘的冰山一角,掌握这些基本的就够了,更多关于它的东西我们会在后面为有兴趣的同志讲解。

\section{排列数公式}
这里的概念来自\textbf{教科书\footnote{即北师大出版的中职学校公共基础课程教材}}上,我们直接在下面引用。

一般地,从$\mathit{n}$个不同元素中任取$m(m \le n)$个元素的所有排列的个数,叫做从$\mathit{n}$个不同元素中任取$\mathit{m}$个元素的\textbf{排列数},用符号$\textbf{A}_{n}^{m}$表示。

很简单,就是有5个水果然后要任选2种水果,并求出共可以组成多少个没有重复组合的两种水果。简而言之就是求在$\textbf{m}$中任取$\textbf{n}$个水果共可以有多少种无重复水果组合。

那怎么求 $\textbf{A}_{n}^{m}$ 呢? 我们知道每一个排列都是从$n$个不同元素中任取$m(m \le n)$个元素,按照顺序排成一列的,所以我们可以把每个排列看成从$n$个不同元素中取$m$次,当然,拿都拿了就不会放回去了。就这样,计算它的排列数可以分$m$步完成。

根据分布计数原理,从$n$个不同元素中取出$m$个元素的排列数即为
\begin{equation}
    \textbf{A}_{n}^{m} = n(n-1)(n-2) \times \cdots{} \times (n-m+1)
\end{equation}
在这个公式中,右边是从正整数$n$开始的$n$个连续的正整数相乘,即从正整数$1$到$n$的连续积,这个连续积就是上面的\textbf{阶乘(详情:\ref{sec:factorial})}。 

所以,从$n$个不同元素的全排列数公式为
\begin{equation*}
    \textbf{A}_{m}^{m}=m!~=m(m-1)(m-2)\times \cdots{}\times 3 \times 2 \times 1
\end{equation*}
这不就是阶乘吗?没错,因为$(n-m)!~=(n-m)(n-m-1)\times (n-m-2) \times \cdots{} 2 \times 1$,所以排列数公式还可以写成
\begin{equation}
    \textbf{A}_{n}^{m}=\frac{n!}{(n-m)!}
\end{equation}

还有一点必须补充,当 $n > m$ 时排列数该怎样算?首先,当 $m>n$ 时,排列数 $\textbf{A}_{n}^m$ 的计算会出现分母为负数阶乘的情况,而负数阶乘在数学中 \textbf{没有定义}。因此,排列数 $\textbf{A}_{n}^m$ 在 $m>n$ 时 无意义。

\section{例题}
{\color{blue} 例一:}{请计算$\textbf{A}_{5}^{2}$}、$\textbf{A}_{11}^{3}$、$\textbf{A}_{n}^{m}$ 的值。

{\color{blue} 解答:}$\textbf{A}_{5}^{2} = 5 \times 4 = 20$, $\textbf{A}_{11}^{3}=11 \times 11 \times 10 \times 9 = 990$, $\textbf{A}_{n}^{n} = n!$

在这里就不给太多例题了,这一章的重点还在后面。下一节就是\textbf{组合},如果上面的东西你学会了那么下一节就不会有任何问题。
