\documentclass{ctexart}
\ctexset{
    section = {
        titleformat = \raggedright,
        name = {,、},
        number = \chinese{section}
    }
}
\usepackage[a4paper,top=2.54cm,bottom=2.54cm,left=3.18cm,right=3.18cm]{geometry}
\usepackage{amsmath,amsfonts,amssymb,amsthm}
\usepackage{tikz}
\usepackage{multirow}
\usepackage{array}
\usepackage{fancyhdr}
\usepackage{lastpage}
\usepackage{enumitem}
\usepackage{caption}

\pagestyle{fancy}
\fancyhf{}
\cfoot{第 \thepage 页~~共 \pageref{LastPage} 页} % 可以使用 \pageref{LastPage}
\renewcommand{\headrulewidth}{0pt}
\renewcommand{\labelenumii}{\Alph{enumii}.}

\begin{document}

%\vspace{1em}
\begin{center}
\textbf{\LARGE 中职数学基础模块}\par
\vspace{8pt}
\textbf{\Large 第二单元测试卷(一)}
\end{center}

\begin{center}
\begin{tabular}{m{0.2\textwidth} m{0.3\textwidth} m{0.25\textwidth}}
     主编:曹一铭 &排版:1805559573@qq.com &测试用时:45分钟
\end{tabular}
\end{center}
\hrule height 1pt
%\noindent
%\rule{\textwidth}{1pt}

\section{单项选择题(每小题6分,共20分)}
\begin{enumerate}
    \item 如果~$a<b$,且~$ac<bc$,那么~( ~~ ).
    \vspace{-2em}
    \begin{center}
    \begin{tabular}{m{0.25\textwidth} m{0.25\textwidth} m{0.25\textwidth} m{0.25\textwidth}}
         A. $c>0$ &B. $c<0$ &C. $c=0$ &D. $c\ge 0$
    \end{tabular}
    \end{center}
    
    \item 不等式$|~x+1~| < 3$的解集为~( ~~ ).
    \vspace{-2em}
    \begin{center}
    \begin{tabular}{m{0.25\textwidth} m{0.25\textwidth} m{0.25\textwidth} m{0.25\textwidth}}
         A. $[-2,~4]$ &B. $[-4,~2]$ &C. $(-2,~4)$ &D. $(-4,~2)$
    \end{tabular}
    \end{center}

    \item 设~$x>0$,则$(x+1)^2$与$x^2 + x + 1$的大小关系是~( ~~ ).
    \vspace{-2em}
    \begin{center}
    \begin{tabular}{m{0.25\textwidth} m{0.25\textwidth} m{0.25\textwidth} m{0.25\textwidth}}
         A. $(x+1)^2 < x^2+x+1$ &B. $(x+1)^2 = x^2+x+1$ &C. $(x+1)^2 > x^2+x+1$ &D. 不能确定
    \end{tabular}
    \end{center}

    \item 设集合 $A = (-4,~2), B=(0,~3)$,则$A \cup B =$~( ~~ ).
    \vspace{-2em}
    \begin{center}
    \begin{tabular}{m{0.25\textwidth} m{0.25\textwidth} m{0.25\textwidth} m{0.25\textwidth}}
         A. $(-4,~3)$ &B. $(0,~2)$ &C. $(-4,~0)$ &D. $(2,~3)$
    \end{tabular}
    \end{center}

\end{enumerate}

\section{填空题(共20分,每题5分)}

    \begin{enumerate}%\setcounter{enumi}{12}
        \item 若 $a>b$,则$a-6$ \underline{~~~~~~} $b-6$.~(填“<”~“>”~“=”)
        
        \item 集合 $\{ x~|~-2\ge x\ge8 \}$ 用区间表示为 \underline{~~~~~~}.
        
        \item 不等式 $(x+2)(x-1) < 0$ 的解集为 \underline{~~~~~~}.

        \item 不等式 $3~|~x~|~ -2 \ge 4$

        \item 设全集 $U = \mathbb{\textbf{R}} $,集合 $A = (-\infty,~2)$,则 $\complement_\textbf{u}A =$ \underline{~~~~~~}.

    \end{enumerate}

\section{解答题(第一题12分,第二、三题各14分,共40分)}

	\begin{enumerate}%\setcounter{enumi}{16}

\item 解不等式组 $\begin{cases}~~|~3-2x~|>1~,\\ ~~\cfrac{x-3}{2} \le \cfrac{-1-x}{3}~.\end{cases}$
  
\vspace{1.5cm}
\newpage

\item  已知二次函数 $y = x^2 - 2x -8$,当 $y < 0$时,求 $x$ 的取值范围.

\vspace{3cm}

\item 用长为~8m~的篱笆围一个花坛,其中一面靠墙,如图所示,花坛的宽为~$x$m~,问当 $x$ 取多少时,可能满足花坛的面积不小于~$6m^2$.

\begin{figure}[h]
     \flushright
     \includegraphics[width=0.30\textwidth]{draw.pdf}
     \captionsetup{justification=raggedleft,singlelinecheck=false}
     \caption{\small{第三题图}}
\end{figure}

\end{enumerate}

\end{document}
