\documentclass[a5paper]{article}
\usepackage{inputenc}
\usepackage[UTF8]{ctex}
\usepackage{geometry}
\geometry{scale=0.80}
\usepackage{amsmath, amssymb}
\usepackage{hyperref}
\hypersetup{
    colorlinks=true, linkcolor=black,
    filecolor=blue, urlcolor=black,
    citecolor=green
}
\pagestyle{empty}

\title{}
\author{}
\date{}

\begin{document}

% \maketitle

\begin{center}
    \textbf{\LARGE 中职数学公式速记手册}
    \vspace{5pt}
\end{center}
\hrule height 1pt

\section{{\large \textbf{集合}}}

\leftline{\textbf{交集的性质}}
\begin{align*}
    & \text{(1)~} A \cap B = B \cap A \\
    & \text{(2)~} \oslash \subseteq A \cap B \subseteq A ~~~ \oslash \subseteq A \cap B \subseteq B \\
    & \text{(3)~} A \cap A = A ~~~ A \cap \oslash = A ~~~ A \cap U = U
\end{align*}

\leftline{\textbf{并集的性质}}
\begin{align*}
    & \text{(1)~} A \cup B = B \cup A \\
    & \text{(2)~} \oslash \subseteq A \cap B \subseteq A \subseteq A ~~~ \oslash \subseteq A \cap B \subseteq B \subseteq A \\
    & \text{(3)~} A \cap A = A ~~~ A \cup \oslash = A ~~~ A \cup U = U
\end{align*}

\leftline{\textbf{补集的性质}}
\begin{align*}
    & A \cup \complement_{U}A = U & A \cap \complement_{U}A = \oslash \\
    & \complement_{U}\oslash = U & \complement_{U}(\complement_{U}A) = A
\end{align*}

% ------
\newpage
\section{{\large \textbf{不等式}}}

\leftline{\textbf{重要性质}}

\begin{align*}
    & \text{(1) 对称性:} a > b \Leftrightarrow b < a \\
    & \text{(2) 传递性:} a>b,b>c \Rightarrow a>c \text{(或} a<b,b<c \Rightarrow a<c~\text{)} \\
    & \text{(3) 基本性质:} 
\end{align*}
\section{{\large \textbf{数列}}}

\leftline{\textbf{数列的前$n$项和}}
\begin{equation*}
    \textbf{S}_n = a_1 + a_2 + a_3 + \cdots + a_n
\end{equation*}
\begin{center}
	\begin{math}
	    a_n =
        \begin{cases}
            \textbf{S}_1,~n = 1 \\
            \textbf{S}_n - \textbf{S}_{n-1},~n \le 2 
        \end{cases}
    \end{math}
\end{center}

\leftline{\textbf{等差数列的通项公式}}
\begin{equation*}
    a_n = a_1 + (n-1)d ~~(n \in \mathbb{N}^*)
\end{equation*}

\leftline{\textbf{等差数列的前$n$项和公式}}
\begin{equation*}
    \textbf{S}_n = \frac{1}{2}n(a_1 + a_n) = na_1 + \frac{1}{2}n(n-1)d
\end{equation*}

\leftline{\textbf{等差数列的性质公式}}
\begin{align*}
    & \text{(1) 若} m+n = p+q \text{,则} a_m + a_n = a_p + a_q \\
    & \text{(2)~} 2m = p+q \text{,则} 2a_m = a_p + a_q \\
    & \text{(3)~} a_n - a_m = (n-m)q ~~(n>m)
\end{align*}

\leftline{\textbf{等比数列的通项公式}}
\begin{equation*}
    a_n = a_1 \cdot q^{n-1}
\end{equation*}

\leftline{\textbf{等比数列的$n$项和公式}}

\begin{center}
    \begin{math}
        \textbf{S}_n =
        \begin{cases}
            \frac{a_1(1-a^n)}{1-q} = \frac{a_1 - a_{n}q}{1-q} ~~(q \ne 1) \\
            na_1 ~~(n > m)
        \end{cases}
    \end{math}
\end{center}

\leftline{\textbf{等比数列的性质式}}

\begin{align*}
    & \text{(1) 若} m+n = p+q \text{,则} a_m \cdot a_n = a_p \cdot a_q \\
    & \text{(2)~} 2m = p+q \text{,则} 2a_m = a_p \cdot a_q \\
    & \text{(3)~} \frac{a_n}{a_m} = q^{n-m} ~~(n>m)
\end{align*}

% ------

\section{{\large \textbf{函数}}}

% ------

\section{{\large \textbf{指数函数与对数函数}}}

% ------

\section{{\large \textbf{三角函数}}}

\leftline{\textbf{同角基本关系式}}

\begin{align*}
    & \text{(1) } \sin^2{\alpha} + \cos^2{\alpha} = 1 \\
    & \text{(2) } \tan{\alpha} = \frac{\sin{\alpha}}{\cos{\alpha}} ~~(\alpha \ne k\pi+\frac{\pi}{2},~k\in{}\mathbb{Z})
\end{align*}

\leftline{\textbf{正弦函数的合角公式}}

\begin{align*}
    & \sin{\alpha + \beta} = \sin{\alpha}\cos{\beta} + \cos{\beta}\sin{\alpha} \\
    & \sin{\alpha - \beta} = \sin{\alpha}\cos{\beta} - \cos{\beta}\sin{\alpha}
\end{align*}

\leftline{\textbf{余弦函数的合角公式}}

\begin{align*}
    & \cos{\alpha + \beta} = \sin{\alpha}\sin{\beta} - \cos{\beta}\cos{\alpha} \\
    & \cos{\alpha - \beta} = \sin{\alpha}\sin{\beta} + \cos{\beta}\cos{\alpha}
\end{align*}

\leftline{\textbf{正切函数的合角公式}}

\begin{align*}
    & \tan{\alpha + \beta} = \frac{\tan{\alpha} + \tan{\beta}}{1-\tan{\alpha}\tan{\beta}} \\
    & \tan{\alpha - \beta} = \frac{\tan{\alpha} - \tan{\beta}}{1+\tan{\alpha}\tan{\beta}}
\end{align*
}

\leftline{\textbf{倍角公式}}

\begin{align*}
    & \sin{2 \alpha} = 2\sin{\alpha}\cos{\beta} \\
    & n\cos{2 \alpha} = \cos^2{\alpha} - \sin^2{\beta} = 2\cos^2{\alpha} - 1 = 1 - 2\sin^2{\alpha} \\
    & \tan{2 \alpha} = \frac{2 \tan{\alpha}}{1 - \tan^2{\alpha}}
\end{align*}
% ------

\section{{\large \textbf{直线与圆的方程}}}

% \leftline{\textbf{}}

% ------

\section{{\large \textbf{命题与充分、必要条件}}}

\leftline{\textbf{命题}}

一般地,把可以判断真假的陈述句叫做\textbf{命题}

\leftline{\textbf{复合命题的形式}}

\begin{itemize}
    \item $p \vee q$ (记作 $p$ 或 $q$)
    \item $p \wedge q$ (记作 $p$ 且 $q$)
    \item $\neg{}p$ (记作 非$p$)
\end{itemize}

\leftline{\textbf{全称量词}}

全称量词的命题形式:
\begin{equation*}
    \forall x \in D,~P(x)
\end{equation*}

全称量词的否定形式:
\begin{equation*}
    \neg(\forall x,P(x)) \equiv \exists{}x, \neg P(x)
\end{equation*}

\leftline{\textbf{存在量词}}

存在量词的命题形式:
\begin{equation*}
    \exists{}x \in D,~P(x)
\end{equation*}

存在量词的否定形式:
\begin{equation*}
    \neg(\exists x,P(x)) \equiv \forall{}x, \neg P(x)
\end{equation*}

% ------

\section{{\large \textbf{排列组合}}}

\leftline{\textbf{排列数公式}}
\begin{equation*}
    \textbf{A}_{n}^{m} = n(n-1)(n-2)\cdots{}(n-m+1)~(n, m \in \mathbb{N}^{*}~\text{且}m \le n)
\end{equation*}

\leftline{\textbf{阶乘}}
\begin{equation*}
    \textbf{A}_{n}^{n} = n! = n(n-1)(n-2) \cdot \cdots{} \cdot 2 \cdot 1
\end{equation*}

\leftline{\textbf{排列数阶乘公式}}
\begin{equation*}
    \textbf{A}_n^{m} = \frac{n!}{(n-m)!}
\end{equation*}

\leftline{\textbf{组合数公式}}
\begin{equation*}
    \textbf{C}_{n}^m = \frac{\textbf{A}_n^{m}}{\textbf{A}_m^{m}} = \frac{n(n-1)(n-2) \cdot \cdots{} \cdot (n-m+1)}{m!}~(n,m \in \mathbb{N}^* \text{且} m \le n)
\end{equation*}

\leftline{\textbf{组合数阶乘公式}}
\begin{equation*}
    \textbf{C}_{n}^m = \frac{\textbf{A}_n^{m}}{\textbf{A}_n^{n}} = \frac{n!}{n! \cdot (n-m)!}~(n,m \in \mathbb{N}^* \text{且} m \le n)
\end{equation*}

\leftline{\textbf{组合数的两个性质}}
\begin{equation*}
    \textbf{C}_n^{m} = \textbf{C}_{n}^{n-m};~~\textbf{C}_{n+1}^{m} = \textbf{C}_n^{m} + \textbf{C}_n^{m-1}~(\textbf{C}_n^{0} = 1)
\end{equation*}

\vspace{5pt}
\hrule height 1pt
\begin{center}
    \textbf{\href{https://github.com/VHS-Survival-Manual/}{VHSS Manual}}
\end{center}

\end{document}
