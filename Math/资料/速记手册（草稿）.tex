\documentclass[a5paper]{article}
\usepackage{inputenc}
\usepackage[UTF8]{ctex}
\usepackage{geometry}
\geometry{scale=0.80}
\usepackage{amsmath, amssymb}
\pagestyle{empty}

\title{}
\author{}
\date{}

\begin{document}

% \maketitle

\begin{center}
    \textbf{\LARGE 中职数学速记手册}
    \vspace{5pt}
\end{center}
\hrule height 1pt

\section{{\large \textbf{集合与逻辑}}}

\leftline{\textbf{交集的性质}}
\begin{align*}
    & \text{(1)~} A \cap B = B \cap A \\
    & \text{(2)~} \oslash \subseteq A \cap B \subseteq A ~~~ \oslash \subseteq A \cap B \subseteq B \\
    & \text{(3)~} A \cap A = A ~~~ A \cap \oslash = A ~~~ A \cap U = U
\end{align*}

\leftline{\textbf{并集的性质}}
\begin{align*}
    & \text{(1)~} A \cup B = B \cup A \\
    & \text{(2)~} \oslash \subseteq A \cap B \subseteq A \subseteq A ~~~ \oslash \subseteq A \cap B \subseteq B \subseteq A \\
    & \text{(3)~} A \cap A = A ~~~ A \cup \oslash = A ~~~ A \cup U = U
\end{align*}

\leftline{\textbf{补集的性质}}
\begin{align*}
    & A \cup \complement_{U}A = U & A \cap \complement_{U}A = \oslash \\
    & \complement_{U}\oslash = U & \complement_{U}(\complement_{U}A) = A
\end{align*}

\section{{\large \textbf{数列}}}

\leftline{\textbf{数列的前$n$项和}}
\begin{equation*}
    \textbf{S}_n = a_1 + a_2 + a_3 + \cdots + a_n
\end{equation*}
\begin{center}
	\begin{math}
	    a_n =
        \begin{cases}
            \textbf{S}_1,~n = 1 \\
            \textbf{S}_n - \textbf{S}_{n-1},~n \le 2 
        \end{cases}
    \end{math}
\end{center}

\leftline{\textbf{等差数列的通项公式}}
\begin{equation*}
    a_n = a_1 + (n-1)d ~~(n \in \mathbb{N}^*)
\end{equation*}

\leftline{\textbf{等差数列的前$n$项和公式}}
\begin{equation*}
    \textbf{S}_n = \frac{1}{2}n(a_1 + a_n) = na_1 + \frac{1}{2}n(n-1)d
\end{equation*}

\leftline{\textbf{等差数列的性质公式}}
\begin{align*}
    & \text{(1) 若} m+n = p+q \text{,则} a_m + a_n = a_p + a_q \\
    & \text{(2)~} 2m = p+q \text{,则} 2a_m = a_p + a_q \\
    & \text{(3)~} a_n - a_m = (n-m)q ~~(n>m)
\end{align*}

\leftline{\textbf{等比数列的通项公式}}
\begin{equation*}
    a_n = a_1 \cdot q^{n-1}
\end{equation*}

\leftline{\textbf{等比数列的$n$项和公式}}

\begin{center}
    \begin{math}
        \textbf{S}_n =
        \begin{cases}
            \frac{a_1(1-a^n)}{1-q} = \frac{a_1 - a_{n}q}{1-q} ~~(q \ne 1) \\
            na_1 ~~(n > m)
        \end{cases}
    \end{math}
\end{center}

\leftline{\textbf{等比数列的性质式}}

\begin{align*}
    & \text{(1) 若} m+n = p+q \text{,则} a_m \cdot a_n = a_p \cdot a_q \\
    & \text{(2)~} 2m = p+q \text{,则} 2a_m = a_p \cdot a_q \\
    & \text{(3)~} \frac{a_n}{a_m} = q^{n-m} ~~(n>m)
\end{align*}

\section{{\large \textbf{排列组合}}}

\leftline{\textbf{排列数公式}}
\begin{equation*}
    \textbf{A}_{n}^{m} = n(n-1)(n-2)\cdots{}(n-m+1)~(n, m \in \mathbb{N}^{*}~\text{且}m \le n)
\end{equation*}

\leftline{\textbf{阶乘}}
\begin{equation*}
    \textbf{A}_{n}^{n} = n! = n(n-1)(n-2) \cdot \cdots{} \cdot 2 \cdot 1
\end{equation*}

\leftline{\textbf{排列数阶乘公式}}
\begin{equation*}
    \textbf{A}_n^{m} = \frac{n!}{(n-m)!}
\end{equation*}

\leftline{\textbf{组合数公式}}
\begin{equation*}
    \textbf{C}_{n}^m = \frac{\textbf{A}_n^{m}}{\textbf{A}_m^{m}} = \frac{n(n-1)(n-2) \cdot \cdots{} \cdot (n-m+1)}{m!}~(n,m \in \mathbb{N}^* \text{且} m \le n)
\end{equation*}

\leftline{\textbf{组合数阶乘公式}}
\begin{equation*}
    \textbf{C}_{n}^m = \frac{\textbf{A}_n^{m}}{\textbf{A}_m^{m}} = \frac{n!}{m! \cdot (n-m)!}~(n,m \in \mathbb{N}^* \text{且} m \le n)
\end{equation*}

\leftline{\textbf{组合数的两个性质}}
\begin{equation*}
    \textbf{C}_n^{m} = \textbf{C}_{n}^{n-m};~~\textbf{C}_{n+1}^{m} = \textbf{C}_n^{m} + \textbf{C}_n^{m-1}~(\textbf{C}_n^{0} = 1)
\end{equation*}

\end{document}
