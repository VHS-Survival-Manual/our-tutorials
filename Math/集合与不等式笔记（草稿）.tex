\documentclass[a5paper]{article}
\usepackage{inputenc}
\usepackage[UTF8]{ctex}
\usepackage{amsmath, amssymb}
\usepackage{tcolorbox,xcolor}
\usepackage{tikz}
\usepackage{ulem}
\usepackage{caption,booktabs}
\usepackage{geometry}
\geometry{scale=0.85}
\usepackage{hyperref}
\hypersetup{
    colorlinks=true, linkcolor=cyan,
    filecolor=blue, urlcolor=red,
    citecolor=green
}

\definecolor{BluePurple1}{RGB}{46, 47, 108}
\definecolor{BluePurple2}{RGB}{139, 128, 172}
\definecolor{BluePurple3}{RGB}{214, 203, 227}

%\title{\Huge \textbf{集合与不等式笔记}}
%\author{\href{https://github.com/Zhou-Yi-Xuan}{easonzhou0801@163.com}}
%\date{\today}

\begin{document}
%\maketitle

\thispagestyle{empty} % 隐藏页码

\begin{tikzpicture}[remember picture, overlay]
% 绘制三拼色背景
\fill[BluePurple1] (current page.north west) rectangle ++(\paperwidth/3, -\paperheight);
\fill[BluePurple2] ([xshift=\paperwidth/3] current page.north west) rectangle ++(\paperwidth/3, -\paperheight);
\fill[BluePurple3] ([xshift=2*\paperwidth/3] current page.north west) rectangle ++(\paperwidth/3, -\paperheight);

% ==== 白色标题背景(全宽度)====
\path [fill=white]
  (current page.north west) ++(0,-7cm)  % 顶部下移3cm
  rectangle
  (current page.north east) ++(0,-10cm); % 总高度3cm

% ==== 标题 ====
\node[align=center, text=BluePurple1, font=\sffamily\bfseries]
  at ([yshift=-4.5cm] current page.north) { % 居中于白色块
    \scalebox{1.7}{\huge 集合与不等式笔记} \\[0.3cm]
    \scalebox{1.1}{\Large \href{https://github.com/Zhou-Yi-Xuan}{easonzhou0801@163.com}}
};
\end{tikzpicture}

\newpage
\tableofcontents
\setcounter{page}{1}

\newpage
\section{\textbf{引言}}

在数学的浩瀚星河中,“集合”与“不等式”如同两把钥匙,一者以简洁的逻辑框架定义万物归属,一者以严谨的符号语言丈量数量边界。集合是数学的基石,从数系的延展到空间的抽象,它以包容的姿态构建起数学世界的秩序;不等式则是探索的工具,无论是函数的变化趋势、方程的解集范围,还是现实问题的优化极限,它用符号的张力划出精确的领域。这本笔记凝结了高一阶段对这两个核心概念的思考与沉淀——从集合的并、交、补运算中领悟逻辑的纯粹,从不等式的变形、求解中体会数学的锋芒。愿这些文字成为你推开数学之门的扉页,在符号与思维的碰撞中,看见理性之美。\footnote{快谢谢\href{https://chat.deepseek.com/}{Deep seek}}

\section{\textbf{集合}}

\subsection{集合元素与常见集合}

一般地,把一些确定的, 可以区分的事物放在一起构成的整体称为集合, 简称集。通常用大写字母表示;集合中的每个确定的对象叫做这个集合的\textbf{元素},通常用小写字母表示。

如果$a$是集合\textbf{A}中的元素,就说\textbf{$a$ 属于 A},记作$a \in A$,读作“a 属于 A”;如果$b$不是集合$A$中的元素,就说\textbf{$b$不属于A},记作$b\notin A$,读作“b不属于A”。

\begin{tcolorbox}[colframe = blue!20, title = {\color{red} 提示}]
    给定一个集合,任何一个对象是否属于这个集合就很明确了。也就是说,给定一个集合,就给定了一个明确的条件,据此可以判断任何一个对象是否属于这个集合。这就说明集合的元素具有\textbf{确定性}。

    例如,“大于10的偶数”可以组成一个集合,将其记为集合$B$,那么集合$B$中的元素就是~12,14,16,18,20,$\cdots$~,则$16 \in B$,$17 \notin B$,$8 \notin B$。

    另外,一个给定集合中的元素不能重复,且再排序上没有顺序要求。也就是说,集合的元素具有\textbf{互异性}和\textbf{无序性}。
\end{tcolorbox}

如果有一个集合,它只包含26位字母(即 a \~ z),是有限个。像这样元素有限的集合,称为\textbf{有限集}。

但如果一个集合它包含$2x-7 \le 0$的所有实数解,集合中元素的个数为无限个,称为\textbf{无限集}。

还有一种特殊的集合,我们会在后面经常见到,这种集合它不包含任何元素,这样的集合被称为\textbf{空集},记作$\oslash$。例如,方程$x^2 +1=0$的实数解组成的集合,它在实数范围内无解,所以形如此类的集合,即为空集。

在集合的表示方法中,常用的有列举法和描述法。列举法,把集合的元素一一列举出来,写在大括号内;描述法,把集合中所有元素的共同特征描述出来。

我们把方程或不等式的所有实数解组成的集合叫作该方程或不等式的\textbf{解集}。由于数轴上的点与实数是一一对应的,所以实数也可以用数轴上的点来表示。

下面是几种特殊集合的表示符号:

$\mathbb{N}$ \text{— 自然数集合(包括0)}

$\mathbb{Z}$ — 整数集合,$\mathbb{Z^+}$ — 正整数集合,$\mathbb{Z^-}$ — 负整数集合

$\mathbb{Q}$ — 有理数集合,$\mathbb{Q^+}$ — 正有理数集合,$\mathbb{Q^-}$ — 负理数集

$\mathbb{R}$ — 实数集合,$\mathbb{R^+}$ — 正实数集合,$\mathbb{R^-}$ — 负实数集合

$\mathbb{C}$ — 复数集合,等等

集合有很多表示方法,这里介绍几种常用的表示方法:

\textbf{列举法(或列元素法)}把集合中的全部元素一一列举出来, 元素之间用逗号“, ”隔开, 并把它们用花括号“\{~\}”括起来. 例如:$A=\{1,2,3,4,5\}$

列元素法一般适合表示元素个数较少的集合. 当集合中元素个数较多时, 如果组成该集合的元素有一定的规律, 也可采用此方法, 此时, 列出部分元素, 当看出组成该集合的其他元素的规律时, 其余元素用“$\cdots$”来表示. 例如:$B=\{1,2,\cdots{},99\}$

此方法也可以表示含有无穷多个元素且元素有一定的规律的集合.

\textbf{描述法}

\textbf{谓词表示法}

\subsection{集合之间的关系}

\subsubsection{子集与真子集}

\subsection{集合之间的运算}

\subsubsection{交集}

\subsubsection{并集}

\subsubsection{差集}

\subsubsection{全集与补集}

\subsection{例题}

\section{不等式}

不等式是数学中一种表示数的大小关系的式子,它通过不等号如“>”、“<”、“$\ge$”、“$'\le$”或“≠”来连接两个表达式,表明它们之间的相对大小。不等式可以分为严格不等式(如 > 和 <,表示一边严格大于或小于另一边)和非严格不等式(如$\ge$ 和 $\ge$,表示一边大于或等于、小于或等于另一边)。不等式不仅包括实数之间的关系,也可以涉及更复杂的数学对象,如向量、矩阵以及函数之间的关系。它们在数学的多个分支,如代数、几何、分析学以及日常生活中都有广泛的应用,比如在估计、优化问题、界限设定和决策分析中。

\subsection{不等式的基本性质}

\textbf{性质一};不等式两边(或减法)同一个数(或式子,不等号方向不变),如果$a>b$,那么$a\pm c>b\pm c$。\textbf{性质二};不等式两边乘(或除)同一个正数,不等号的方向不变,如果$a>b,c>0$,那么$ac>bc$(或$\frac{a}{c}>\frac{b}{c}$)。\textbf{性质三};

\subsection{区间}

\subsection{一元二次不等式}

\subsection{含绝对值的不等式}

\subsection{不等式的应用}

\end{document}
