\documentclass[a5paper]{article}
\usepackage{inputenc}
\usepackage[UTF8]{ctex}
\usepackage{amsmath, amssymb}
\usepackage{tcolorbox,xcolor}
\usepackage{tikz}
\usepackage{ulem}
\usepackage{caption,booktabs}
\usepackage{geometry}
\geometry{scale=0.85}
\usepackage{hyperref}
\hypersetup{
    colorlinks=true, linkcolor=cyan,
    filecolor=blue, urlcolor=red,
    citecolor=green
}

\definecolor{BluePurple1}{RGB}{46, 47, 108}   % 深海蓝紫
\definecolor{BluePurple2}{RGB}{139, 128, 172} % 迷雾灰紫
\definecolor{BluePurple3}{RGB}{214, 203, 227} % 晨曦浅紫

%\title{\Huge \textbf{集合与不等式笔记}}
%\author{\href{https://github.com/Zhou-Yi-Xuan}{easonzhou0801@163.com}}
%\date{\today}

\begin{document}
%\maketitle

\thispagestyle{empty} % 隐藏页码

\begin{tikzpicture}[remember picture, overlay]
% 绘制三拼色背景
\fill[BluePurple1] (current page.north west) rectangle ++(\paperwidth/3, -\paperheight);
\fill[BluePurple2] ([xshift=\paperwidth/3] current page.north west) rectangle ++(\paperwidth/3, -\paperheight);
\fill[BluePurple3] ([xshift=2*\paperwidth/3] current page.north west) rectangle ++(\paperwidth/3, -\paperheight);

% ==== 白色标题背景(全宽度)====
\path [fill=white] 
  (current page.north west) ++(0,-7cm)  % 顶部下移3cm
  rectangle 
  (current page.north east) ++(0,-10cm); % 总高度3cm

% ==== 标题 ====
\node[align=center, text=BluePurple1, font=\sffamily\bfseries] 
  at ([yshift=-4.5cm] current page.north) { % 居中于白色块
    \scalebox{1.7}{\huge 集合与不等式笔记} \\[0.3cm]
    \scalebox{1.1}{\Large \href{https://github.com/Zhou-Yi-Xuan}{easonzhou0801@163.com}}
};
\end{tikzpicture}

\newpage
\tableofcontents
\setcounter{page}{1}

\newpage
\section{\textbf{引言}}

在数学的浩瀚星河中,“集合”与“不等式”如同两把钥匙,一者以简洁的逻辑框架定义万物归属,一者以严谨的符号语言丈量数量边界。集合是数学的基石,从数系的延展到空间的抽象,它以包容的姿态构建起数学世界的秩序;不等式则是探索的工具,无论是函数的变化趋势、方程的解集范围,还是现实问题的优化极限,它用符号的张力划出精确的领域。这本笔记凝结了高一阶段对这两个核心概念的思考与沉淀——从集合的并、交、补运算中领悟逻辑的纯粹,从不等式的变形、求解中体会数学的锋芒。愿这些文字成为你推开数学之门的扉页,在符号与思维的碰撞中,看见理性之美。\footnote{快谢谢\href{https://chat.deepseek.com/}{Deep seek}}

本笔记引用了许多网上资料,感谢无私奉献的开源精神!厚德载物的青春万岁!

\section{\textbf{集合}}

\subsection{集合元素与常见集合}

由一些确定的对象组成的整体就称为\textbf{集合}(简称\textbf{集}),通常用大写字母表示;集合中的每个确定的对象叫做这个集合的\textbf{元素},通常用小写字母表示。

如果$a$是集合\textbf{A}中的元素,就说\textbf{$a$ 属于 A},记作$a \in A$,读作“a 属于 A”;如果$b$不是集合$A$中的元素,就说\textbf{$b$不属于A},记作$b\notin A$,读作“b不属于A”。

\begin{tcolorbox}[colframe = blue!20, title = {\color{red} 提示}]
    给定一个集合,任何一个对象是否属于这个集合就很明确了。也就是说,给定一个集合,就给定了一个明确的条件,据此可以判断任何一个对象是否属于这个集合。这就说明集合的元素具有\textbf{确定性}。

    例如,“大于10的偶数”可以组成一个集合,将其记为集合$B$,那么集合$B$中的元素就是~12,14,16,18,20,$\cdots$~,则$16 \in B$,$17 \notin B$,$8 \notin B$。

    另外,一个给定集合中的元素不能重复,且再排序上没有顺序要求。也就是说,集合的元素具有\textbf{互异性}和\textbf{无序性}。
\end{tcolorbox}

如果有一个集合,它只包含26位字母(即 a \~ z),是有限个。像这样元素有限的集合,称为\textbf{有限集}。

但如果一个集合它包含$2x-7 \le 0$的所有实数解,集合中元素的个数为无限个,称为\textbf{无限集}。

还有一种特殊的集合,我们会在后面经常见到,这种集合它不包含任何元素,这样的集合被称为\textbf{空集},记作$\oslash$。例如,方程$x^2 +1=0$的实数解组成的集合,它在实数范围内无解,所以形如此类的集合,即为空集。

在集合的表示方法中,常用的有列举法和描述法。列举法,把集合的元素一一列举出来,写在大括号内;描述法,把集合中所有元素的共同特征描述出来。

我们把方程或不等式的所有实数解组成的集合叫作该方程或不等式的\textbf{解集}。由于数轴上的点与实数是一一对应的,所以实数也可以用数轴上的点来表示。

\subsection{集合之间的关系}

\section{不等式}

\subsection{不等式的基本性质}

\subsection{区间}

\subsection{一元二次不等式}

\subsection{含绝对值的不等式}

\subsection{不等式的应用}

\end{document}
