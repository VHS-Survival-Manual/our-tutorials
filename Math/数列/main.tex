\documentclass[a5paper]{article}
\usepackage{inputenc}
\usepackage[UTF8]{ctex}
\usepackage{amsmath, amssymb}
\usepackage{tcolorbox,xcolor}
\usepackage{tikz}
\usepackage{caption,booktabs,ulem}
\usepackage{geometry}
\geometry{scale=0.85}
\usepackage{hyperref}
\hypersetup{
    colorlinks=true, linkcolor=cyan,
    filecolor=blue, urlcolor=red,
    citecolor=green
}

\definecolor{BluePurple1}{RGB}{46, 47, 108}
\definecolor{BluePurple2}{RGB}{139, 128, 172}
\definecolor{BluePurple3}{RGB}{214, 203, 227}

\begin{document}
% \maketitle

\begin{tikzpicture}[remember picture, overlay]
% 绘制三拼色背景
\fill[BluePurple1] (current page.north west) rectangle ++(\paperwidth/3, -\paperheight);
\fill[BluePurple2] ([xshift=\paperwidth/3] current page.north west) rectangle ++(\paperwidth/3, -\paperheight);
\fill[BluePurple3] ([xshift=2*\paperwidth/3] current page.north west) rectangle ++(\paperwidth/3, -\paperheight);

% ==== 白色标题背景(全宽度)====
\path [fill=white] 
  (current page.north west) ++(0,-7cm)  % 顶部下移3cm
  rectangle 
  (current page.north east) ++(0,-10cm); % 总高度3cm

% ==== 标题 ====
\node[align=center, text=BluePurple1, font=\sffamily\bfseries] 
  at ([yshift=-4.5cm] current page.north) { % 居中于白色块
    \scalebox{2.4}{\huge 数列笔记} \\[0.3cm]
    \scalebox{1.5}{\Large \href{https://github.com/Zhou-Yi-Xuan}{easonzhou0801@163.com}}
};
\end{tikzpicture}
\thispagestyle{empty} % 隐藏页码
\newpage

%\tableofcontents
%\setcounter{page}{1}

\part{\textbf{引论}}

数列,这数学中的一串璀璨明珠,自古以来便是探索秩序与模式的窗口。它以最朴素的形式——一连串数字,
隐藏着无限的奥秘与变化,从等差的简洁到等比的韵律,乃至更复杂的递归序列,每一种数列都像是数学世界的一首诗,
叙述着数与数之间的和谐与冲突。在数学的长河中,数列不仅是初等数学的基石,更是通往高等数学殿堂的桥梁,
它在现代科学、工程、金融等领域中的应用无处不在,从计算复利到描述物理现象,其重要性不言而喻。
通过研究数列的性质、求和、极限,我们不仅能够洞察数学的内在美,还能培养逻辑思维与抽象思考的能力,
为解决实际问题提供强有力的工具。让我们一起踏上这段探索数列之美的旅程,揭开它背后的数学真理,
感受数学世界的无限魅力。\footnote{来自~\href{https://chat.deepseek.com/}{deep seek}}

什么是数列?假如我们有10个数字它们分别是:$1,~2, \cdots~ 9,~10$,把它们放在一个“集合”当中,
简而言之数列就是按一定顺序排列的一列数,记作$\{a_n\}$,其中每个数称为数列的\textbf{项},
$a_n$表示第$n$项(通项)。上面我们所举的例子就是一个\textbf{有穷数列},在这里我们就顺便引出后面的概念。
根据数列中项数是否为有限值,可以将数列分为两类:若有限,例如 $1,~2,~3,~4$则称为
\textbf{有穷数列(finite sequence)};若无限,例如 $1,~2,~3,~4,~\cdots$ 则称为\textbf{无穷数列(infinite sequence)}。虽然有穷数列与无穷数列在项数上有显著差异,但它们之间的联系
也十分紧密:就像前面的例子那样,有穷数列可以看作是无穷数列的前部分截取。因此,在研究规律时,
有穷数列的许多特性也适用于无穷数列,反之亦然。在高中阶段,只研究有穷数列。
\footnote{感谢~\href{https://wuli.wiki/online/HsSeFu.html}{小时百科}}

将一些数按照一定的次序排列成一列,称为\textbf{数列(number sequence)},通常记作:
\begin{equation*}
    a_1,~a_2,~a_3,~\cdots ,~a_n,~\cdots
\end{equation*}
一般记作:$\{a_n\}$,字母$n$可以替换为其他字母。其中$n$是自然数,数列中的每一个数
被称为该数列的\textbf{项(term)}。

\newpage
我们再补充写关于项的概念:
\begin{enumerate}
    \item $a_1$ 被称为该数列的\textbf{首项(first term)}
    \item $a_n$ 表示数列的第 $n$ 项,也被称为\textbf{通项(general term)}
    \item 数列中包含项的个数称为\textbf{项数}。
\end{enumerate}

在有穷数列中,其最后一项通常被称为\textbf{末项(last term)},这一项标志着数列的结束。
如果从倒序的角度观察数列,首项和末项的位置相互调换。例如,$1,2,3,4$的末项是$4$,但如果倒序观察,
则数列变为$4,3,2,1$,此时的末项变为原数列的首项$1$。这种倒序视角在研究数列特性时提供了一种新的方式,
特别是在对称性或逆向推导等问题中具有独特的价值。无论是正序还是倒序,数列的规律依然保持不变,
只是观察的顺序发生了变化。

如果数列$a_n$的第$n$项$a_n$ 与$n$的关系可以用一个表达式为$a_n = f(n)$之间的关系可以用一个表达式
表示为 ,那么这个表达式被称为该数列的\textbf{通项公式(general term formula)}。

尽管数列的通项公式非常实用,但由于数列的离散特性,在推导和使用时必须特别关注
\textbf{边界条件(boundary conditions)},例如首项 $a_1$ 或末项 $a_n$ 等特定项,
是否满足通项公式。如果通项公式无法涵盖所有项或忽略特定边界条件,就可能导致表达不完整或错误。
相比之下,函数在许多场合下也需要考虑边界,但由于函数的连续性,边界问题通常表现为平滑的过渡,
不需要过多关注离散点的特殊性。对于数列,只有确保通项公式对每一项都成立,才能准确描述数列的规律。

不过关于上面那段话我们就不深入解答了,这个笔记只解决高中的数列问题,对于第三部分我们只是为了给有兴趣的同学所
编写的拓展部分。



\part{\textbf{排列}}

\section{排列问题}

什么是排列问题?我们先列举一个例子来给大家说明一下。假设水果篮中有苹果、香蕉、橙子、葡萄和芒果
五种不同的水果,若从中选取两种水果按特定顺序排列(例如「苹果→香蕉」和「香蕉→苹果」视为两种不同的
排列),则不同的顺序会直接影响排列结果。例如,从5种水果中选2种排列时,
排列数为$\textbf{A}_{5}^2 = 5 \times 4 = 20$种,,下面是它们的两个性质:

\begin{enumerate}
    \item \textbf{有序性:}如同水果在果盘中的位置(首位放苹果或首位放葡萄)直接影响最终形态;
    \item \textbf{无序性:}一种水果不能同时占据两个位置(例如苹果不能既在首位又在次位)。
\end{enumerate}

而组合问题则类似「水果盲盒」:只需关注选中的水果种类(如苹果和葡萄),不关心被选中的顺序(苹果先放或葡萄先放均视为同一种组合)。大家可以在本子上画几个点

\subsection{排列问题概括}

我们把被选取的对象(上面水果中的任意一个)叫做元素。那么上面的问题就是从五个不同的元素中任选两个元素,然后按照一定的顺序排成一列,从$\mathit{n}$个不同元素中任取$m(m \le n)$个元素,按照一定的顺序排成一列,叫作从$\mathit{n}$个不同元素中任取$\mathit{m}$个元素的一个\textbf{排列}。

\subsection{阶乘}\label{sec:factorial}
阶乘,是数学中的一个基本概念,表示为$n!$,它定义为所有小于及等于$n$的正整数的乘积。具体来说,对于一个非负整数$n$,其阶乘$n!$计算方式是从1开始,一直乘到$n$,即:
\begin{equation*}
    n! = 1 \times 2 \times 3 \times \cdots \times n
\end{equation*}
或者
\begin{equation*}
    n! = n \times (n-1) \times (n-2) \times \cdots \times 2 \times 1
\end{equation*}

在大多数情况下十位数以上的的阶乘往往需要使用计算器来辅助计算,如:$12!$,你能快速地算出结果吗?,答案是\textbf{479001600},我相信没有人能算的这么快,一般建议大家熟记前十位阶乘的结果,这就够了。

{\color{red} 注意:}我们定义\textbf{$0! = 1$(零的阶乘等于一)},当然这只是关于阶乘的冰山一角,掌握这些基本的就够了,更多关于它的东西我们会在后面为有兴趣的同志讲解。

\section{排列数公式}
这里的概念来自\textbf{教科书\footnote{即北师大出版的中职学校公共基础课程教材}}上,我们直接在下面引用。

一般地,从$\mathit{n}$个不同元素中任取$m(m \le n)$个元素的所有排列的个数,叫做从$\mathit{n}$个不同元素中任取$\mathit{m}$个元素的\textbf{排列数},用符号$\textbf{A}_{n}^{m}$表示。

很简单,就是有5个水果然后要任选2种水果,并求出共可以组成多少个没有重复组合的两种水果。简而言之就是求在$\textbf{m}$中任取$\textbf{n}$个水果共可以有多少种无重复水果组合。

那怎么求 $\textbf{A}_{n}^{m}$ 呢? 我们知道每一个排列都是从$n$个不同元素中任取$m(m \le n)$个元素,按照顺序排成一列的,所以我们可以把每个排列看成从$n$个不同元素中取$m$次,当然,拿都拿了就不会放回去了。就这样,计算它的排列数可以分$m$步完成。

根据分布计数原理,从$n$个不同元素中取出$m$个元素的排列数即为
\begin{equation}
    \textbf{A}_{n}^{m} = n(n-1)(n-2) \times \cdots{} \times (n-m+1)
\end{equation}
在这个公式中,右边是从正整数$n$开始的$n$个连续的正整数相乘,即从正整数$1$到$n$的连续积,这个连续积就是上面的\textbf{阶乘(详情:\ref{sec:factorial})}。 

所以,从$n$个不同元素的全排列数公式为
\begin{equation*}
    \textbf{A}_{m}^{m}=m!~=m(m-1)(m-2)\times \cdots{}\times 3 \times 2 \times 1
\end{equation*}
这不就是阶乘吗?没错,因为$(n-m)!~=(n-m)(n-m-1)\times (n-m-2) \times \cdots{} 2 \times 1$,所以排列数公式还可以写成
\begin{equation}
    \textbf{A}_{n}^{m}=\frac{n!}{(n-m)!}
\end{equation}

还有一点必须补充,当 $n > m$ 时排列数该怎样算?首先,当 $m>n$ 时,排列数 $\textbf{A}_{n}^m$ 的计算会出现分母为负数阶乘的情况,而负数阶乘在数学中 \textbf{没有定义}。因此,排列数 $\textbf{A}_{n}^m$ 在 $m>n$ 时 无意义。

\section{例题}
{\color{blue} 例一:}{请计算$\textbf{A}_{5}^{2}$}、$\textbf{A}_{11}^{3}$、$\textbf{A}_{n}^{m}$ 的值。

{\color{blue} 解答:}$\textbf{A}_{5}^{2} = 5 \times 4 = 20$, $\textbf{A}_{11}^{3}=11 \times 11 \times 10 \times 9 = 990$, $\textbf{A}_{n}^{n} = n!$

在这里就不给太多例题了,这一章的重点还在后面。下一节就是\textbf{组合},如果上面的东西你学会了那么下一节就不会有任何问题。


% \part{\textbf{等比数列}}


% \part{\textbf{数列极限}}


\end{document}
