\documentclass[a5paper]{article}
\usepackage{inputenc}
\usepackage[UTF8]{ctex}
\usepackage{amsmath, amssymb}
\usepackage{tcolorbox,xcolor}
\usepackage{tikz}
\usepackage{ulem}
\usepackage{caption,booktabs}
\usepackage{geometry}
\geometry{scale=0.85}
\usepackage{hyperref}
\hypersetup{
    colorlinks=true, linkcolor=cyan,
    filecolor=blue, urlcolor=red,
    citecolor=green
}

\definecolor{BluePurple1}{RGB}{46, 47, 108}
\definecolor{BluePurple2}{RGB}{139, 128, 172}
\definecolor{BluePurple3}{RGB}{214, 203, 227}

\begin{document}
%\maketitle

\thispagestyle{empty} % 隐藏页码

\begin{tikzpicture}[remember picture, overlay]
% 绘制三拼色背景
\fill[BluePurple1] (current page.north west) rectangle ++(\paperwidth/3, -\paperheight);
\fill[BluePurple2] ([xshift=\paperwidth/3] current page.north west) rectangle ++(\paperwidth/3, -\paperheight);
\fill[BluePurple3] ([xshift=2*\paperwidth/3] current page.north west) rectangle ++(\paperwidth/3, -\paperheight);

% ==== 白色标题背景(全宽度)====
\path [fill=white] 
  (current page.north west) ++(0,-7cm)  % 顶部下移3cm
  rectangle 
  (current page.north east) ++(0,-10cm); % 总高度3cm

% ==== 标题 ====
\node[align=center, text=BluePurple1, font=\sffamily\bfseries] 
  at ([yshift=-4.5cm] current page.north) { % 居中于白色块
    \scalebox{1.7}{\huge 集合与不等式笔记} \\[0.3cm]
    \scalebox{1.1}{\Large \href{https://github.com/Zhou-Yi-Xuan}{easonzhou0801@163.com}}
};
\end{tikzpicture}

\newpage
\tableofcontents
\setcounter{page}{1}

\newpage
\section{\textbf{引言}}

在数学的浩瀚星河中,“集合”与“不等式”如同两把钥匙,一者以简洁的逻辑框架定义万物归属,一者以严谨的符号语言丈量数量边界。集合是数学的基石,从数系的延展到空间的抽象,它以包容的姿态构建起数学世界的秩序;不等式则是探索的工具,无论是函数的变化趋势、方程的解集范围,还是现实问题的优化极限,它用符号的张力划出精确的领域。这本笔记凝结了高一阶段对这两个核心概念的思考与沉淀——从集合的并、交、补运算中领悟逻辑的纯粹,从不等式的变形、求解中体会数学的锋芒。愿这些文字成为你推开数学之门的扉页,在符号与思维的碰撞中,看见理性之美。\footnote{快谢谢\href{https://chat.deepseek.com/}{Deep seek}}

% 集合
\section{\textbf{计数原理}}

{\large 分类计数原理}~~完成一件事,有$n$类办法,在第1类办法中有$m_1$种不同的办法,在第$n$类办法中有$m_n$种不同的办法,那么完成这件事共有$\mathbf{N}m_{1}+m_{2}+\cdots{}+m_{n}$种不同的办法。

{\large 分步计数原理}~~完成一件事,有$n$类办法,在第1类办法中有$m_1$种不同的办法,在第$n$类办法中有$m_n$种不同的办法,那么完成这件事共有$\mathbf{N}m_{1}\cdot{}m_{2}\cdot{}\cdots{}\cdot{}m_{n}$种不同的办法。

\subsection{例题}

{\color{blue} 例一:}乔布斯有4个苹果和3个菠萝,库克要选择一种水果,问有多少种选择

{\color{blue} 解答:}$4+3=7$

{\color{blue} 例二:}乔布斯有4个苹果和3个菠萝以及5个柚子,库克要选择两种水果,问有多少种选择

{\color{blue} 解答:}$4\times{}3\times{}5=60$

\section{\textbf{排列}}

\subsection{排列问题}

我们先从\textbf{排列问题}开始学习。

假设水果篮中有苹果、香蕉、橙子、葡萄和芒果五种不同的水果,若从中选取两种水果按特定顺序排列(例如「苹果→香蕉」和「香蕉→苹果」视为两种不同的排列),则不同的顺序会直接影响排列结果。例如,从5种水果中选2种排列时,排列数为$\textbf{P}(5,~2) = 5 \times 4 = 20$种

\textbf{——}这类似于将水果按不同顺序摆成果盘(如香蕉在前或苹果在前),每种顺序代表独特的外观。

\begin{enumerate}
    \item \textbf{有序性:}如同水果在果盘中的位置(首位放苹果或首位放葡萄)直接影响最终形态;
    \item \textbf{无序性:}一种水果不能同时占据两个位置(例如苹果不能既在首位又在次位)。
\end{enumerate}

而组合问题则类似「水果盲盒」:只需关注选中的水果种类(如苹果和葡萄),不关心被选中的顺序(苹果先放或葡萄先放均视为同一种组合)。

\subsubsection{排列问题概括}

我们把被选取的对象(上面的水果中的任意一个)叫做元素。那么上面的问题就是从五个不同的元素中任选两个元素,然后按照一定的顺序排成一列,从$\mathit{n}$个不同元素中任取$m(m \le n)$个元素,按照一定的顺序排成一列,叫作从$\mathit{n}$个不同元素中任取$\mathit{m}$个元素的一个\textbf{排列}。

\subsection{排列数公式}
这里的概念来自\textbf{教科书\footnote{即北师大出版的中职学校公共基础课程教材}}上,我们直接在下面引用。

一般地,从$\mathit{n}$个不同元素中任取$m(m \le n)$个元素的所有排列的个数,叫做从$\mathit{n}$个不同元素中任取$\mathit{m}$个元素的\textbf{排列数},用符号$\textbf{A}_{n}^{m}$表示。

很简单,就是有5个水果然后要任选2种水果,并求出共可以组成多少个没有重复组合的两种水果。简而言之就是求在$\textbf{m}$中任取$\textbf{n}$个水果共可以有多少种无重复水果组合。

但在这之前我们来了解下什么是\textbf{阶乘}

\subsubsection{阶乘}\label{sec:factorial}
阶乘,是数学中的一个基本概念,表示为$n!$,它定义为所有小于及等于$n$的正整数的乘积。具体来说,对于一个非负整数$n$,其阶乘$n!$计算方式是从1开始,一直乘到$n$,即:
\begin{equation*}
    n! = 1 \times 2 \times 3 \times \cdots \times n
\end{equation*}

没错,又是~AI~生成。但这个真的很简单,相信聪明的你很快就理解了。在大多数情况下十位数的阶乘往往需要使用计算器来辅助计算,如:$12!$,你能快速地算出结果吗?我觉得你不是电脑,答案是\textbf{479001600},我相信没有人能算的这么快,一般建议大家熟记前十位阶乘的结果,这就够了。

\begin{flushleft}
	{\color{red} 注意:}我们定义\textbf{$0! = 1$(零的阶乘等于一)}
\end{flushleft}

\noindent\hfil$*$\hfil$*$\hfil$*$\hfil \\

\textbf{那怎么求 $\textbf{A}_{n}^{m}$ 呢?} 

我们知道每一个排列都是从$n$个不同元素中任取$m(m \le n)$个元素,按照顺序排成一列的,所以我们可以把每个排列看成从$n$个不同元素中取$m$次,当然,拿都拿了就不会放回去了。就这样,计算它的排列数可以分$m$步完成。

根据分布计数原理,从$n$个不同元素中取出$m$个元素的排列数即为
\begin{equation*}
    \textbf{A}_{n}^{m} = n(n-1)(n-2)*\cdots{}*(n-m+1)
\end{equation*}
在这个公式中,右边是从正整数$n$开始的$n$个连续的正整数相乘,即从正整数$1$到$n$的连续积,这个连续积就是上面的\textbf{阶乘(详情:\ref{sec:factorial})}。

所以,从$n$个不同元素的全排列数公式为
\begin{equation*}
    \textbf{A}_{n}^{n}=n!~=n(n-1)(n-2)*\cdots{}*3*2*1
\end{equation*}
因为$(n-m)!~=(n-m)(n-m-1)*(n-m-2)*\cdots{}*3*2*1$,所以排列数公式还可以写成
\begin{equation}
    \textbf{A}_{n}^{m}=\frac{n!}{(n-m)!}
\end{equation}

还记得上面我们关于\textbf{阶乘}的定义吗?$0!~=~1$,所以,当$n=m$时,$(n-m)!~=0!~=1$,上述公式依然成立

\subsubsection{例题}
{\color{blue} 例一:}{请计算$\textbf{A}_{5}^{2}$}~与~$\textbf{A}_{11}^{3}$ 的值。

{\color{blue} 解答:}$\textbf{A}_{5}^{2} = 5 \times 4 = 20$,$\textbf{A}_{11}^{3}=11 \times 11 \times 10 \times 9 = 990$

在这里就不给太多例题了,这一章的重点还在后面。下一节就是\textbf{组合},如果上面的东西你学会了那么下一节就不会有任何问题。


% 不等式
\section{\textbf{组合}}

\subsection{组合问题}

首先要我们要搞懂怎样判断哪些是组合问题、哪些是排列问题。还是以上面的\underline{水果篮子}为例。
\begin{itemize}
    \item 从五种水果中任选三种,分别送给\textbf{三}个不同的人,请问共有多少种方案?
    \item 从五种水果中任选两种,然后送给\textbf{一}个人,共可组成多少种方案?
    \item 从2,3,5,7,11~这5个数中任取两个数,共可组成多少个不同的分数?\sout{(不用水果了)}
    \item 从2,3,5,7,11~这5个数中任取两个数,共可组成多少个不同的真分数?
\end{itemize}

\leftline{{\color{blue} {\large 解答}}}
\begin{itemize}
    \item 这是\textbf{排列问题};$n$个水果分给$m$个人($n \ne m$),因为每人所得的水果不一样。
    \item 这是\textbf{组合问题};水果都给一个人了,不管选取顺序如何结果都相同。%($\textbf{A}_{5}^{1}$)
    \item 这是\textbf{排列问题};若选取到5和7,它们既可以组合成$\frac{7}{5}$也可以组合成$\frac{5}{7}$,很明显有两种组合方法(分母与分子的位置可以互换。没有唯一性)
    \item 这是\textbf{组合问题};还是跟上面一样选取到5和7,但要求是组合成\textbf{真分数\footnote{什么是真分数?!简单来说就是分母大于分子的分数}},所以就只有一种组合。
\end{itemize}

\subsubsection{组合问题概括}
从$n$个不同元素中任取$m(m \le n)$个元素,组成一组,叫做从$n$个不同元素中取出$m$个元素的一个\textbf{组合}。

如果两个组合中的元素完全一致,无论元素选取的顺序如何,那么它们就是相同的组合。只有当两个组合中的元素不完全相同时,它们才是不同的组合。这里是不是很像我们在高一学习的\textbf{集合},还记得集合中的概念吗?

它们两个的共同点:
\begin{enumerate}
    \item \textbf{无序性的本质:} 组合问题中的元素选择(例如从5种水果中选2种)与集合中的元素存储,均不依赖顺序。例如集合中的$\{ a,~b\}$ 与 $\{ b,~a\}$ 它们的本质是相同的。
    \item \textbf{唯一性的映射:} 集合要求元素互异,而组合问题中选取的元素也同样如此。
\end{enumerate}
研究从$n$个不同元素中任取$m(m \le n)$个元素的所有组合的个数,这类计数问题叫做\textbf{组合问题}。

\subsection{组合数公式}

排列数是在$m$个数中任选$n$个数,然后排列出它们的具体数量,而组合数则是求所有排列数的个数,也就是将每个单独的排列数中包含的个数相加。下面就是它的抽象概括。

从$n$个不同元素中任取$m(m \le n)$个元素的所有组合的个数,叫做从$n$个不同元素任取$m$个元素的\textbf{组合数},用符号$\textbf{C}_{n}^{m}$表示。

一般地,求出$n$个不同元素中任取$m$个元素的排列组合数$\textbf{A}_{n}^m$,可以分两步完成。
\begin{enumerate}
    \item 求出从$n$个不同的元素中任取$m$个元素的组合数,有$\textbf{C}_n^{m}$个;
    \item 对每一个组合中的$m$个元素进行全排列,其排列数均为$\textbf{A}_{m}^{m}$。
\end{enumerate}
根据分步计数原理,可得
\begin{equation}
    \textbf{A}_{m}^{n} = \textbf{C}_{n}^{m} \times \textbf{A}_{m}^{m}
\end{equation}
因此,组合数的计算公式为
\begin{equation}
    \textbf{C}_{n}^{m}=\frac{\textbf{A}_{n}^{m}}{\textbf{A}_{m}^{m}}=\frac{n(n-1)(n-2)\cdot{}\cdots{}\cdot{}(n-m+1)}{m!}=\frac{n!}{m!(n-m)!}
\end{equation}
其中$m \in \mathbf{N}$,$n \in \mathbf{N}_{+}$,且$m \le n$。这个公式就是\textbf{组合数公式}。

下面再给出它们的一些性质

\begin{equation}
    \textbf{C}_{n}^{m} + \textbf{C}_{n}^{m+1} = \textbf{C}_{n+1}^{m+1}
\end{equation}

\subsubsection{组合数的不同写法}

在研究离散数学\footnote{即组合数学}时我们常常会碰到一些不同的写法,如:$\binom{n}{m}$、$\textbf{P}(n,~k)$以及上面我们所用的$\textbf{C}_{n}^m$等,在我们继续深入理解组合数学时往往会因为这些符号的写法而感到迷惑,这里就列出它们之间的关系:
\begin{equation*}
    \textbf{C}_{n}^{r} = \binom{n}{r} = \textbf{P}(n,~r)
\end{equation*}

上面的等式则为组合数的三种写法,在众多高等数学教科书中更常用的是圆括号表达组合数$\binom{n}{m}$,而我们的笔记中所使用的则是苏式表达$\textbf{C}_{n}^m$。我们发现该等式的后面还有一个大写$\textbf{P}$的表达方式,实际上这是老旧的表达方式从本质上讲,$\textbf{P}(n,~r)$与$\textbf{C}_{n}^m$是一样的,$r$即为$m$,所以用$P$表达法\footnote{暂时找不到标准称呼}我们可以得到另一个等式:
\begin{equation*}
    \frac{\textbf{P}(n,~r)}{r!} = \textbf{A}_{n}^r
\end{equation*}

让我们把$r$换成$m$这样更好理解。我们来验证这个等式。
\begin{equation*}
    \begin{aligned}
        &\textbf{A}_{n}^m = \textbf{C}_{m}^n \cdot \textbf{A}_{m}^{m} \\
        &\textbf{A}_{n}^m = \textbf{P}(n,~m) \cdot m! \\
        &\frac{\textbf{A}_{n}^m}{m!} = \textbf{P}(n,~m)
    \end{aligned}
\end{equation*}

证毕!就是这样,在后面的"二项式定理\ref{sec:factorial}"中我们还会看到它。

\subsection{组合数的两个性质}

组合数有如下的两种性质。
\begin{flushleft}{\color{blue} 性质一}\end{flushleft}
\begin{equation}
    \textbf{C}_{n}^{m}=\textbf{C}_{n}^{n-m}
\end{equation}
\begin{flushleft}
	{\color{blue} 证明}
\end{flushleft}
\begin{equation*}
    \begin{aligned}
        &\textbf{C}_{n}^{m} = \frac{n!}{m!(n-m)!} \\
        &\textbf{C}_{n}^{n-m} = \frac{n!}{(n-m)![n-(n-m)]!} = \frac{n!}{(n-m)!m!}
    \end{aligned}
\end{equation*}

所以$\textbf{C}_{n}^{m}=\textbf{C}_{n}^{m-n}$。

我们页可以这样理解,从$n$个不同的元素中取出$m$个元素组成一组后,剩下的$(n-m)$个元素自然也组成一组,即每取出$m$个元素都有唯一的$(n-m)$个元素之对应。所以,从$n$个不同的元素中取出$m$个元素的组合数一定与从$n$个不同的元素中取出$(n-m)$个元素的组合数相等。

组合数的第二个性质
\begin{flushleft}{\color{blue} 性质二}\end{flushleft}
\begin{equation}
    \textbf{C}_{n}^{m}+\textbf{C}_{n}^{m-1}=\textbf{C}_{n+1}^{m}
\end{equation}
\begin{flushleft}
	{\color{blue} 证明}
\end{flushleft}
\begin{equation*}
    \begin{aligned}
        \textbf{C}_{n}^{m}+\textbf{C}_{n}^{m-1} &= \frac{n!}{m!(n-m)!}+\frac{n!}{(m-1)![n-(m-1)]!} \\
        &= \frac{n!(n-m+1)}{m!(n-m+1)!}+\frac{n!m}{m!(n-m+1)!} \\
        &= \frac{n![(n-m+1)+m]}{m!(n-m+1)!} \\
        &= \frac{(n+m)!}{m!(n-m+1)!}
        &= \textbf{C}_{n+1}^{m}
    \end{aligned}
\end{equation*}

所以$\textbf{C}_{n}^{m}+\textbf{C}_{n}^{m-1}=\textbf{C}_{n+1}^{m}$。

我们也可以直观地理解以上的性质,从$n$个苹果和1个菠萝中任取$m$个水果,可以用组合数$\textbf{C}_{n+1}^{m}$表示,也可以将其分为两类:一类分到的$m$个水果都是苹果,则有$\textbf{C}_{n}^{m}$种分法;另一类抽到的$m$个水果中有一个是菠萝,则有$\textbf{C}_{n}^{m-1}$,故共有$\textbf{C}_{n}^{m}+\textbf{C}_{n}^{m-1}$种取法,所以$\textbf{C}_{n}^{m}+\textbf{C}_{n}^{m-1}=\textbf{C}_{n+1}^{m}$。

\subsubsection{例题}

{\color{blue} 例一:}{请计算$\textbf{C}_{5}^{2}$}~与~$\textbf{C}_{9}^{3}$ 的值。

{\color{blue} 解答:}$\textbf{C}_{5}^{2}=\frac{5\times{}4}{2!}=10$,$\textbf{C}_{6}^{3}=\frac{\textbf{A}_{6}^{3}}{\textbf{A}_{n}^{m}}=\frac{6\times{}5\times{}4}{3!} = 15$

{\color{blue} 例二:}{请计算$\textbf{C}_{100}^{98}$}~与~$\textbf{C}_{9}^{3}+\textbf{C}_{9}^{4}$ 的值。

{\color{blue} 解答:}
\begin{equation*}
    % 考虑使用表格
    \begin{aligned}
        \textbf{C}_{100}^{98} &= \textbf{C}_{100}^{2} \\
        &=  \frac{\textbf{A}_{100}^{2}}{2!} \\
        &= 4950
    \end{aligned}
\end{equation*}

{\color{blue} 例三:}{请计算 $\textbf{C}_{7}^{2}+\textbf{C}_{7}^{3}+\textbf{C}_{8}^{4}+\textbf{C}_{9}^{5}+\textbf{C}_{10}^{6}$ 的值。

{\color{blue} 解答:}
\begin{equation*}
    \begin{aligned}
        \textbf{C}_{7}^{2}+\textbf{C}_{7}^{3}+\textbf{C}_{8}^{4}+\textbf{C}_{9}^{5}+\textbf{C}_{10}^{6} &= \frac{\textbf{A}_{7}^{2}}{2!}+\frac{\textbf{A}_{7}^{3}}{3!}+\frac{\textbf{A}_{8}^{4}}{4!}+\frac{\textbf{A}_{9}^{5}}{5!}+\frac{\textbf{A}_{10}^6}{6!}\\
        &= \frac{42}{2}+\frac{215}{6}+\frac{1680}{24}+\frac{15120}{120}+\frac{151200}{720}\\
        &= 462
    \end{aligned}
\end{equation*}


% 高观点下的集合与不等式
% \section{\textbf{高观点下的集合与不等式}}

% 拓展模块

\end{document}