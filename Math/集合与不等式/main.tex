\documentclass[a5paper]{article}
\usepackage{inputenc}
\usepackage[UTF8]{ctex}
\usepackage{amsmath, amssymb}
\usepackage{tcolorbox,xcolor}
\usepackage{tikz}
\usepackage{ulem}
\usepackage{caption,booktabs}
\usepackage{geometry}
\geometry{scale=0.85}
\usepackage{hyperref}
\hypersetup{
    colorlinks=true, linkcolor=cyan,
    filecolor=blue, urlcolor=red,
    citecolor=green
}

\definecolor{BluePurple1}{RGB}{46, 47, 108}
\definecolor{BluePurple2}{RGB}{139, 128, 172}
\definecolor{BluePurple3}{RGB}{214, 203, 227}

\begin{document}
%\maketitle

\thispagestyle{empty} % 隐藏页码

\begin{tikzpicture}[remember picture, overlay]
% 绘制三拼色背景
\fill[BluePurple1] (current page.north west) rectangle ++(\paperwidth/3, -\paperheight);
\fill[BluePurple2] ([xshift=\paperwidth/3] current page.north west) rectangle ++(\paperwidth/3, -\paperheight);
\fill[BluePurple3] ([xshift=2*\paperwidth/3] current page.north west) rectangle ++(\paperwidth/3, -\paperheight);

% ==== 白色标题背景(全宽度)====
\path [fill=white] 
  (current page.north west) ++(0,-7cm)  % 顶部下移3cm
  rectangle 
  (current page.north east) ++(0,-10cm); % 总高度3cm

% ==== 标题 ====
\node[align=center, text=BluePurple1, font=\sffamily\bfseries] 
  at ([yshift=-4.5cm] current page.north) { % 居中于白色块
    \scalebox{1.7}{\huge 集合与不等式笔记} \\[0.3cm]
    \scalebox{1.1}{\Large \href{https://github.com/Zhou-Yi-Xuan}{easonzhou0801@163.com}}
};
\end{tikzpicture}

\newpage
\tableofcontents
\setcounter{page}{1}

\newpage
\section{\textbf{引言}}

在数学的浩瀚星河中,“集合”与“不等式”如同两把钥匙,一者以简洁的逻辑框架定义万物归属,一者以严谨的符号语言丈量数量边界。集合是数学的基石,从数系的延展到空间的抽象,它以包容的姿态构建起数学世界的秩序;不等式则是探索的工具,无论是函数的变化趋势、方程的解集范围,还是现实问题的优化极限,它用符号的张力划出精确的领域。这本笔记凝结了高一阶段对这两个核心概念的思考与沉淀——从集合的并、交、补运算中领悟逻辑的纯粹,从不等式的变形、求解中体会数学的锋芒。愿这些文字成为你推开数学之门的扉页,在符号与思维的碰撞中,看见理性之美。\footnote{快谢谢\href{https://chat.deepseek.com/}{Deep seek}}

% 集合
\section{\textbf{集合}}

\subsection{集合元素与常见集合}

一般地,把一些确定的, 可以区分的事物放在一起构成的整体称为集合, 简称集。通常用大写字母表示;集合中的每个确定的对象叫做这个集合的\textbf{元素},通常用小写字母表示。

如果$a$是集合\textbf{A}中的元素,就说\textbf{$a$ 属于 A},记作$a \in A$,读作“a 属于 A”;如果$b$不是集合$A$中的元素,就说\textbf{$b$不属于A},记作$b\notin A$,读作“b不属于A”。

\begin{tcolorbox}[colframe = blue!20, title = {\color{red} 提示}]
    给定一个集合,任何一个对象是否属于这个集合就很明确了。也就是说,给定一个集合,就给定了一个明确的条件,据此可以判断任何一个对象是否属于这个集合。这就说明集合的元素具有\textbf{确定性}。

    例如,“大于10的偶数”可以组成一个集合,将其记为集合$B$,那么集合$B$中的元素就是~12,14,16,18,20,$\cdots$~,则$16 \in B$,$17 \notin B$,$8 \notin B$。

    另外,一个给定集合中的元素不能重复,且再排序上没有顺序要求。也就是说,集合的元素具有\textbf{互异性}和\textbf{无序性}。
\end{tcolorbox}

如果有一个集合,它只包含26位字母(即 a \~ z),是有限个。像这样元素有限的集合,称为\textbf{有限集}。

但如果一个集合它包含$2x-7 \le 0$的所有实数解,集合中元素的个数为无限个,称为\textbf{无限集}。

还有一种特殊的集合,我们会在后面经常见到,这种集合它不包含任何元素,这样的集合被称为\textbf{空集},记作$\oslash$。例如,方程$x^2 +1=0$的实数解组成的集合,它在实数范围内无解,所以形如此类的集合,即为空集。

在集合的表示方法中,常用的有列举法和描述法。列举法,把集合的元素一一列举出来,写在大括号内;描述法,把集合中所有元素的共同特征描述出来。

我们把方程或不等式的所有实数解组成的集合叫作该方程或不等式的\textbf{解集}。由于数轴上的点与实数是一一对应的,所以实数也可以用数轴上的点来表示。

下面是几种特殊集合的表示符号:

$\mathbb{N}$ \text{— 自然数集合(包括0)}

$\mathbb{Z}$ — 整数集合,$\mathbb{Z^+}$ — 正整数集合,$\mathbb{Z^-}$ — 负整数集合

$\mathbb{Q}$ — 有理数集合,$\mathbb{Q^+}$ — 正有理数集合,$\mathbb{Q^-}$ — 负理数集

$\mathbb{R}$ — 实数集合,$\mathbb{R^+}$ — 正实数集合,$\mathbb{R^-}$ — 负实数集合

$\mathbb{C}$ — 复数集合,等等

集合有很多表示方法,这里介绍几种常用的表示方法:

\textbf{列举法(或列元素法)}把集合中的全部元素一一列举出来, 元素之间用逗号“, ”隔开, 并把它们用花括号“\{~\}”括起来. 例如:$A=\{1,2,3,4,5\}$

列元素法一般适合表示元素个数较少的集合. 当集合中元素个数较多时, 如果组成该集合的元素有一定的规律, 也可采用此方法, 此时, 列出部分元素, 当看出组成该集合的其他元素的规律时, 其余元素用“$\cdots$”来表示. 例如:$B=\{1,2,\cdots{},99\}$

此方法也可以表示含有无穷多个元素且元素有一定的规律的集合.

\textbf{描述法}

\textbf{谓词表示法}

\subsection{集合之间的关系}

\subsubsection{子集与真子集}

\newpage
\subsection{集合之间的运算}

\subsubsection{交集与并集}

原本这里是两个单独的小结,不过我还是把它们放在了一起,这两个概念还是放在一起更容易理解。

设$A,~B$是两个集合,由既属于集合$A$又属于集合$B$的元素所构成的集合称为$A$与$B$的\textbf{交集}。记作$A\cap{}B$,读作$A$交$B$。即$A\cap{}B=\{x \in A \text{且} x \in B\}$,其主要性质:

$A\cap{}B=B\cap{}A;~A\cup{}A=A;~A\cap{}\oslash = \oslash$

\begin{figure}[ht]
     \flushright
     \includegraphics[width=0.30\textwidth]{image/set_1.pdf}
     \captionsetup{justification=raggedleft,singlelinecheck=false}
     \caption{\small{交集}}
\end{figure}

设$A,~B$是两个集合,由所有属于集合$A$或者属于集合$B$的元素所构成的集合称为$A$与$B$的\textbf{并集}。记作$A\cup{}B$,读作$A$并$B$。即$A\cup{}B=\{x \in A \text{或} x \in B\}$,其主要性质:

$A\cup{}B=B\cup{}A;~A\cup{}A=A;~A\cup{}\oslash = A$

\begin{figure}[ht]
     \flushright
     \includegraphics[width=0.30\textwidth]{image/union_set.pdf}
     \captionsetup{justification=raggedleft,singlelinecheck=false}
     \caption{\small{并集}}
\end{figure}

\subsubsection{差集}

对于集合$A \text{与} B$,由集合$A$中不属于集合$B$的元素所组成的集合即为\textbf{差集}。记作$A \setminus B = \{x~|~x \in A \text{且} x \notin B\}$其主要性质:

\begin{equation*}
    \begin{aligned}
        & A \setminus B \ne B \setminus A;~(A \setminus B)\setminus C \ne A \setminus (B \setminus C); \\
        & ~A \setminus \oslash = A~\&~\oslash \setminus A = \oslash~\&~ A \setminus A = \oslash
    \end{aligned}
\end{equation*}

再补充一条性质:若$U$为全集,则$A \setminus B = A \cap B^c (\complement_{U}B)$,其中$\complement_{U}B=U \setminus B$

\subsubsection{全集与补集}

一般地,如果一个集合含有我们研究问题中所涉及的所有元素,那么这个集合就称这个集合为全集,通常记为$U$或$I$。月有阴晴圆缺,一般地,设$U$为全集,
且$A \subseteq U$由$U$中所有不属于$A$的元素组成的集合,称为集合$A$在$U$中的补集,记作$\complement_{U}A~\text{或者}~A^c~\text{再或者}~A'$\footnote{不过对于这种写法我很不推荐,$A'$的补集运算依赖全集$U'$,需明确上下文中的全集定义},即$\complement_{U}A=\{x~|~x \in U,~\text{且}x \notin A\}$其主要性质:

$A \cap \complement_{U}A = \oslash;~A \cup \complement_{U}A = U;~\complement_{U}(\complement_{U}A)=A$

再来说明下关于补集符号的区别。首先是这种——$\complement_{U}A$,这种是最标准的补集符号被广泛应用于各大教材,也是我们在学习集时看见几率最大的符号,而~$A^c$~则是属于更高等的教材中所使用的

\subsection{例题}


% 不等式
\section{\textbf{组合}}

\subsection{组合问题}

首先要我们要搞懂怎样判断哪些是组合问题、哪些是排列问题。还是以上面的\underline{水果篮子}为例。
\begin{itemize}
    \item 从五种水果中任选三种,分别送给\textbf{三}个不同的人,请问共有多少种方案?
    \item 从五种水果中任选两种,然后送给\textbf{一}个人,共可组成多少种方案?
    \item 从2,3,5,7,11~这5个数中任取两个数,共可组成多少个不同的分数?\sout{(不用水果了)}
    \item 从2,3,5,7,11~这5个数中任取两个数,共可组成多少个不同的真分数?
\end{itemize}

\leftline{{\color{blue} {\large 解答}}}
\begin{itemize}
    \item 这是\textbf{排列问题};$n$个水果分给$m$个人($n \ne m$),因为每人所得的水果不一样。
    \item 这是\textbf{组合问题};水果都给一个人了,不管选取顺序如何结果都相同。%($\textbf{A}_{5}^{1}$)
    \item 这是\textbf{排列问题};若选取到5和7,它们既可以组合成$\frac{7}{5}$也可以组合成$\frac{5}{7}$,很明显有两种组合方法(分母与分子的位置可以互换。没有唯一性)
    \item 这是\textbf{组合问题};还是跟上面一样选取到5和7,但要求是组合成\textbf{真分数\footnote{什么是真分数?!简单来说就是分母大于分子的分数}},所以就只有一种组合。
\end{itemize}

\subsubsection{组合问题概括}
从$n$个不同元素中任取$m(m \le n)$个元素,组成一组,叫做从$n$个不同元素中取出$m$个元素的一个\textbf{组合}。

如果两个组合中的元素完全一致,无论元素选取的顺序如何,那么它们就是相同的组合。只有当两个组合中的元素不完全相同时,它们才是不同的组合。这里是不是很像我们在高一学习的\textbf{集合},还记得集合中的概念吗?

它们两个的共同点:
\begin{enumerate}
    \item \textbf{无序性的本质:} 组合问题中的元素选择(例如从5种水果中选2种)与集合中的元素存储,均不依赖顺序。例如集合中的$\{ a,~b\}$ 与 $\{ b,~a\}$ 它们的本质是相同的。
    \item \textbf{唯一性的映射:} 集合要求元素互异,而组合问题中选取的元素也同样如此。
\end{enumerate}
研究从$n$个不同元素中任取$m(m \le n)$个元素的所有组合的个数,这类计数问题叫做\textbf{组合问题}。

\subsection{组合数公式}

排列数是在$m$个数中任选$n$个数,然后排列出它们的具体数量,而组合数则是求所有排列数的个数,也就是将每个单独的排列数中包含的个数相加。下面就是它的抽象概括。

从$n$个不同元素中任取$m(m \le n)$个元素的所有组合的个数,叫做从$n$个不同元素任取$m$个元素的\textbf{组合数},用符号$\textbf{C}_{n}^{m}$表示。

一般地,求出$n$个不同元素中任取$m$个元素的排列组合数$\textbf{A}_{n}^m$,可以分两步完成。
\begin{enumerate}
    \item 求出从$n$个不同的元素中任取$m$个元素的组合数,有$\textbf{C}_n^{m}$个;
    \item 对每一个组合中的$m$个元素进行全排列,其排列数均为$\textbf{A}_{m}^{m}$。
\end{enumerate}
根据分步计数原理,可得
\begin{equation}
    \textbf{A}_{m}^{n} = \textbf{C}_{n}^{m} \times \textbf{A}_{m}^{m}
\end{equation}
因此,组合数的计算公式为
\begin{equation}
    \textbf{C}_{n}^{m}=\frac{\textbf{A}_{n}^{m}}{\textbf{A}_{m}^{m}}=\frac{n(n-1)(n-2)\cdot{}\cdots{}\cdot{}(n-m+1)}{m!}=\frac{n!}{m!(n-m)!}
\end{equation}
其中$m \in \mathbf{N}$,$n \in \mathbf{N}_{+}$,且$m \le n$。这个公式就是\textbf{组合数公式}。

下面再给出它们的一些性质

\begin{equation}
    \textbf{C}_{n}^{m} + \textbf{C}_{n}^{m+1} = \textbf{C}_{n+1}^{m+1}
\end{equation}

\subsubsection{组合数的不同写法}

在研究离散数学\footnote{即组合数学}时我们常常会碰到一些不同的写法,如:$\binom{n}{m}$、$\textbf{P}(n,~k)$以及上面我们所用的$\textbf{C}_{n}^m$等,在我们继续深入理解组合数学时往往会因为这些符号的写法而感到迷惑,这里就列出它们之间的关系:
\begin{equation*}
    \textbf{C}_{n}^{r} = \binom{n}{r} = \textbf{P}(n,~r)
\end{equation*}

上面的等式则为组合数的三种写法,在众多高等数学教科书中更常用的是圆括号表达组合数$\binom{n}{m}$,而我们的笔记中所使用的则是苏式表达$\textbf{C}_{n}^m$。我们发现该等式的后面还有一个大写$\textbf{P}$的表达方式,实际上这是老旧的表达方式从本质上讲,$\textbf{P}(n,~r)$与$\textbf{C}_{n}^m$是一样的,$r$即为$m$,所以用$P$表达法\footnote{暂时找不到标准称呼}我们可以得到另一个等式:
\begin{equation*}
    \frac{\textbf{P}(n,~r)}{r!} = \textbf{A}_{n}^r
\end{equation*}

让我们把$r$换成$m$这样更好理解。我们来验证这个等式。
\begin{equation*}
    \begin{aligned}
        &\textbf{A}_{n}^m = \textbf{C}_{m}^n \cdot \textbf{A}_{m}^{m} \\
        &\textbf{A}_{n}^m = \textbf{P}(n,~m) \cdot m! \\
        &\frac{\textbf{A}_{n}^m}{m!} = \textbf{P}(n,~m)
    \end{aligned}
\end{equation*}

证毕!就是这样,在后面的"二项式定理\ref{sec:factorial}"中我们还会看到它。

\subsection{组合数的两个性质}

组合数有如下的两种性质。
\begin{flushleft}{\color{blue} 性质一}\end{flushleft}
\begin{equation}
    \textbf{C}_{n}^{m}=\textbf{C}_{n}^{n-m}
\end{equation}
\begin{flushleft}
	{\color{blue} 证明}
\end{flushleft}
\begin{equation*}
    \begin{aligned}
        &\textbf{C}_{n}^{m} = \frac{n!}{m!(n-m)!} \\
        &\textbf{C}_{n}^{n-m} = \frac{n!}{(n-m)![n-(n-m)]!} = \frac{n!}{(n-m)!m!}
    \end{aligned}
\end{equation*}

所以$\textbf{C}_{n}^{m}=\textbf{C}_{n}^{m-n}$。

我们页可以这样理解,从$n$个不同的元素中取出$m$个元素组成一组后,剩下的$(n-m)$个元素自然也组成一组,即每取出$m$个元素都有唯一的$(n-m)$个元素之对应。所以,从$n$个不同的元素中取出$m$个元素的组合数一定与从$n$个不同的元素中取出$(n-m)$个元素的组合数相等。

组合数的第二个性质
\begin{flushleft}{\color{blue} 性质二}\end{flushleft}
\begin{equation}
    \textbf{C}_{n}^{m}+\textbf{C}_{n}^{m-1}=\textbf{C}_{n+1}^{m}
\end{equation}
\begin{flushleft}
	{\color{blue} 证明}
\end{flushleft}
\begin{equation*}
    \begin{aligned}
        \textbf{C}_{n}^{m}+\textbf{C}_{n}^{m-1} &= \frac{n!}{m!(n-m)!}+\frac{n!}{(m-1)![n-(m-1)]!} \\
        &= \frac{n!(n-m+1)}{m!(n-m+1)!}+\frac{n!m}{m!(n-m+1)!} \\
        &= \frac{n![(n-m+1)+m]}{m!(n-m+1)!} \\
        &= \frac{(n+m)!}{m!(n-m+1)!}
        &= \textbf{C}_{n+1}^{m}
    \end{aligned}
\end{equation*}

所以$\textbf{C}_{n}^{m}+\textbf{C}_{n}^{m-1}=\textbf{C}_{n+1}^{m}$。

我们也可以直观地理解以上的性质,从$n$个苹果和1个菠萝中任取$m$个水果,可以用组合数$\textbf{C}_{n+1}^{m}$表示,也可以将其分为两类:一类分到的$m$个水果都是苹果,则有$\textbf{C}_{n}^{m}$种分法;另一类抽到的$m$个水果中有一个是菠萝,则有$\textbf{C}_{n}^{m-1}$,故共有$\textbf{C}_{n}^{m}+\textbf{C}_{n}^{m-1}$种取法,所以$\textbf{C}_{n}^{m}+\textbf{C}_{n}^{m-1}=\textbf{C}_{n+1}^{m}$。

\subsubsection{例题}

{\color{blue} 例一:}{请计算$\textbf{C}_{5}^{2}$}~与~$\textbf{C}_{9}^{3}$ 的值。

{\color{blue} 解答:}$\textbf{C}_{5}^{2}=\frac{5\times{}4}{2!}=10$,$\textbf{C}_{6}^{3}=\frac{\textbf{A}_{6}^{3}}{\textbf{A}_{n}^{m}}=\frac{6\times{}5\times{}4}{3!} = 15$

{\color{blue} 例二:}{请计算$\textbf{C}_{100}^{98}$}~与~$\textbf{C}_{9}^{3}+\textbf{C}_{9}^{4}$ 的值。

{\color{blue} 解答:}
\begin{equation*}
    % 考虑使用表格
    \begin{aligned}
        \textbf{C}_{100}^{98} &= \textbf{C}_{100}^{2} \\
        &=  \frac{\textbf{A}_{100}^{2}}{2!} \\
        &= 4950
    \end{aligned}
\end{equation*}

{\color{blue} 例三:}{请计算 $\textbf{C}_{7}^{2}+\textbf{C}_{7}^{3}+\textbf{C}_{8}^{4}+\textbf{C}_{9}^{5}+\textbf{C}_{10}^{6}$ 的值。

{\color{blue} 解答:}
\begin{equation*}
    \begin{aligned}
        \textbf{C}_{7}^{2}+\textbf{C}_{7}^{3}+\textbf{C}_{8}^{4}+\textbf{C}_{9}^{5}+\textbf{C}_{10}^{6} &= \frac{\textbf{A}_{7}^{2}}{2!}+\frac{\textbf{A}_{7}^{3}}{3!}+\frac{\textbf{A}_{8}^{4}}{4!}+\frac{\textbf{A}_{9}^{5}}{5!}+\frac{\textbf{A}_{10}^6}{6!}\\
        &= \frac{42}{2}+\frac{215}{6}+\frac{1680}{24}+\frac{15120}{120}+\frac{151200}{720}\\
        &= 462
    \end{aligned}
\end{equation*}


% 高观点下的集合与不等式
% \section{\textbf{高观点下的集合与不等式}}

% 拓展模块

\end{document}