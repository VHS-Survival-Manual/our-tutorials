\section{\textbf{不等式}}

不等式是数学中一种表示数的大小关系的式子,它通过不等号如“>”、“<”、\newline
“$\ge$”、“$'\le$”或“$\ne$”来连接两个表达式,表明它们之间的相对大小。不等式可以分为严格不等式(如 > 和 <,表示一边严格大于或小于另一边)和非严格不等式(如$\ge$ 和 $\ge$,表示一边大于或等于、小于或等于另一边)。不等式不仅包括实数之间的关系,也可以涉及更复杂的数学对象,如向量、矩阵以及函数之间的关系。它们在数学的多个分支,如代数、几何、分析学以及日常生活中都有广泛的应用,比如在估计、优化问题、界限设定和决策分析中。

\subsection{不等式的基本性质}

\textbf{性质一};不等式两边(或减法)同一个数(或式子,不等号方向不变),如果$a>b$,那么$a\pm c>b\pm c$。\textbf{性质二};不等式两边乘(或除)同一个正数,不等号的方向不变,如果$a>b,c>0$,那么$ac>bc$(或$\frac{a}{c}>\frac{b}{c}$)。\textbf{性质三};

\subsubsection{做差比较}

\subsection{区间}

\subsection{一元二次不等式}

\subsubsection{概念}

\subsubsection{解法}

\subsection{含绝对值的不等式}

\subsubsection{基本解法}

\subsubsection{特殊解法}

\subsection{例题}
