\documentclass{report}
\usepackage[utf8]{inputenc}
\usepackage{ctex}
\usepackage{lipsum}
\usepackage[hooks]{tcolorbox}
\usepackage{hyperref}
\hypersetup {
    colorlinks=true, linkcolor=cyan, filecolor=blue,
    urlcolor=red, citecolor=green
}
\newtcbox{\Inline}[1][red]{
	on line, arc = 0pt, outer arc = 0pt,
    colback = #1!10!white, colframe = #1!50!black,
    boxsep = 0pt, left = 1pt, right = 1pt, top = 2pt, bottom = 2pt,
    boxrule = 0pt, bottomrule = 1pt, toprule = 1pt
}

\begin{document}

\title{\fontsize{56pt}\selectfont{\textbf{简单 Git}}}
\author{\fontsize{48pt}\selectfont{\href{https://gtihub.com/VHS-Survival-Manual/}{VHSS Manual - }}}
\date{\today}

\maketitle

\newpage
\section*{设置Git}

\subsection*{下载 Git}

在浏览器中打开 \textbf{Git} 的官方\href{https://git-scm.com/downloads}{下载链接},
在里面有很详细的安装方式,这里不再介绍。

\begin{tcolorbox}
git --version
\end{tcolorbox}

查看它的版本,如果该步骤没有问题说明 \textbf{Git} 已经成功安装。

\subsection*{添加 Github SSH密钥}

首先创建你的\txetbf{SSH}密钥,打开你的终端(或命令提示符):
\begin{tcolorbox}
ssh-keygen -t rsa -C "<Your email>"
\end{tcolorbox}

\Inline[yellow]{注意!这里必须确保你的系统开启了 \textbf{SSH} 功能!}

回车该命令,然后一路回车,此方法对大多数系统有效。完成后继续输入下面的命令。
打开你的家目录并进入 \Inline[green]{.ssh} 目录,然后复制密钥。

\begin{tcolorbox}[title = {\textbf{Windows}}]
cd .ssh\texttt{\char92}
notepad id\_rsa.pub
\end{tcolorbox}

\begin{tcolorbox}[title = {\textbf{Linux or MacOS}}]
cd \$HOME/.ssh/ \par
cat id\_rsa.pub
\end{tcolorbox}

进入\textbf{\href{https://github.com/settings/}{Github 设置}},
找到 \Inline[blue]{“SSH and GPG keys”} 设置选项,然后添加新的密钥(请参考网页上的说明)。

完成后再次打开终端并输入以下命令进行测试

\begin{tcolorbox}
ssh -T git@github.com
\end{tcolorbox}

如果显示通过则密钥设置成功,如果失败请复制错误原因自行搜索。

\section*{开始工作}

\subsection*{创建储存库}

首先设置你的用户名以及邮箱:

\begin{tcolorbox}
git config --global user.name "<Your user name>"
git config --global user.email "<Your email>"
\end{tcolorbox}

完成后进入你的\Inline[blue]{目录},然后完成初始化操作。

\begin{tcolorbox}
git init
\end{tcolorbox}

如果你不想让\textbf{Git}的初始化库的名称为\texbf{master},那就看看下面的命令以更改设置。
这里我们假设你要将\texbf{master}改为\textbf{main}

\begin{tcolorbox}
git config --global init.defaultbranch main
\end{tcolorbox}

\subsection*{分支}

创建一个名叫“dev” 的分支并切换

\begin{tcolorbox}
git checkout -b dev || git branch dev
\end{tcolorbox}

切换分支使用:

\begin{tcolorbox}
git checkout dev
\end{tcolorbox}

编写完成后,要把dev分支合并到主分支,先切换到主分支再merge:

\begin{tcolorbox}
git checkout master \\
git merge dev
\end{tcolorbox}

最后删除已合并的分支请使用\Inline[green]{-d},删除为合并过的请使用\Inline[green]{-D}参数

\begin{tcolorbox}
git branch -d dev \\
git branch -D feature-vulcan
\end{tcolorbox}

\subsection*{本地提交}

\end{document}

